
%  next 2 lines pull in the NASA.cls file, put options between []
\documentclass[]             % options: RDPonly, coveronly, nocover
{aiaa-tc}                       %   plus standard article class options


% ----------- start extension to sub-sub-sub section capability --------- %
\usepackage{titlesec}
\usepackage{hyperref}

\titleclass{\subsubsubsection}{straight}[\subsection]

\newcounter{subsubsubsection}[subsubsection]
\renewcommand\thesubsubsubsection{\thesubsubsection.\arabic{subsubsubsection}}
\renewcommand\theparagraph{\thesubsubsubsection.\arabic{paragraph}} % optional; useful if paragraphs are to be numbered

\titleformat{\subsubsubsection}
  {\normalfont\normalsize\bfseries}{\thesubsubsubsection}{1em}{}
\titlespacing*{\subsubsubsection}
{0pt}{3.25ex plus 1ex minus .2ex}{1.5ex plus .2ex}

\makeatletter
\renewcommand\paragraph{\@startsection{paragraph}{5}{\z@}%
  {3.25ex \@plus1ex \@minus.2ex}%
  {-1em}%
  {\normalfont\normalsize\bfseries}}
\renewcommand\subparagraph{\@startsection{subparagraph}{6}{\parindent}%
  {3.25ex \@plus1ex \@minus .2ex}%
  {-1em}%
  {\normalfont\normalsize\bfseries}}
\def\toclevel@subsubsubsection{4}
\def\toclevel@paragraph{5}
\def\toclevel@paragraph{6}
\def\l@subsubsubsection{\@dottedtocline{4}{7em}{4em}}
\def\l@paragraph{\@dottedtocline{5}{10em}{5em}}
\def\l@subparagraph{\@dottedtocline{6}{14em}{6em}}
\makeatother

\setcounter{secnumdepth}{4}
\setcounter{tocdepth}{4}
% ----------- ------------------------------------------------- --------- %

\title{Conceptual Feasibility Study of the Hyperloop Vehicle for Next-Generation Transport}

\author{
  Kenneth Decker,\thanks{Aerospace Engineer, University of Arizona}
  \and Jeffrey Chin,\thanks{Aerospace Engineer, Propulsion Systems Analysis Branch, Mail Stop 5-11, AIAA Member}
  \and Andi Peng,\thanks{Cognitive Science, Global Affairs, Yale University}
  \and Colin Summers,\thanks{Chemical Engineer, Computer Engineer, University of Washington-Seattle}
  \and Golda Nguyen,\thanks{Mechanical Engineer, Georgia Institute of Technology}
  \and Andrew Oberlander,\thanks{Mechanical Engineer, Brown University}
  \and Gazi Sakib,\thanks{Physics, Mechanical Engineer, Stony Brook University}
  \and Nathan Sharifrazi,\thanks{Aerospace Engineer, Mechanical Engineer, University of California-Irvine}
  \and Justin Gray,\thanksibid{2}
  \and Christopher Heath,\thanksibid{2}
  \and Robert Falck, \thanks{Aerospace Engineer, Mission Architecture and Analysis Branch, Mail Stop 162-2, AIAA Member}
  \and
  {\normalsize\itshape
  NASA Glenn Research Center, Cleveland, OH}
}

% Define commands to assure consistent treatment throughout document
\newcommand{\eqnref}[1]{(\ref{#1})}
\newcommand{\class}[1]{\texttt{#1}}
\newcommand{\package}[1]{\texttt{#1}}
\newcommand{\file}[1]{\texttt{#1}}
\newcommand{\BibTeX}{\textsc{Bib}\TeX}

\setlength{\abovecaptionskip}{0pt}
\setlength{\belowcaptionskip}{0pt}

\usepackage{minted} %syntax highlighting
\usepackage{color} %syntax highlighting
\usepackage{siunitx} % correct unit typesetting
\usepackage{graphicx}
% \usepackage{changepage}
\usepackage{amsmath} %math equations
\usepackage{hyperref} %hyperlinks
\usepackage{courier} %courier font for variable names
\usepackage{cleveref} %section references (must come after hyperref)
\usepackage[all]{hypcap} %figure references point to top of figure (must come after hyperref)
\usepackage{nomencl} %nomenclature
% \usepackage{subfiles} %split sections into separate docs
\usepackage{import}
\usepackage{setspace} %Pro­vides sup­port for set­ting the spac­ing be­tween lines in a doc­u­ment
\usepackage{wrapfig} %Al­lows fig­ures or ta­bles to have text wrapped around them.
\usepackage{caption} %The cap­tion pack­age pro­vides many ways to cus­tomise the cap­tions in float­ing en­vi­ron­ments like fig­ure and ta­ble
\usepackage{subcaption} % Allows subfigures/captions
% \usepackage{lscape} %Mod­i­fies the mar­gins and ro­tates the page con­tents but not the page num­ber.
\usepackage{appendix}
\usepackage{listings} %Code
\usepackage[section]{placeins} %De­fines a \FloatBar­rier com­mand, be­yond which floats may not pass;
\usepackage[superscript]{cite} %The pack­age sup­ports com­pressed, sorted lists of nu­mer­i­cal ci­ta­tions
\usepackage{esdiff} %The pack­age makes writ­ing deriva­tives very easy.
\usepackage{bm} %The bm pack­age de­fines a com­mand \bm which makes its ar­gu­ment bold. (math)
\usepackage{tabulary} %wider table
\usepackage{gensymb} %degree symbol
\usepackage{letltxmacro}
\usepackage{booktabs} % en­hances the qual­ity of ta­bles
\usepackage{tikz} % used to generate XDSM key
\usepackage[trackold]{trackchanges} % used to annotate and collaborate on document editing

\addeditor{Colin} % adds editor to trackchanges

\lstset{frame=single}
\newcommand{\txt}{\textrm}

\newcommand{\cent}{{\mathrm{c}\mkern-6.5mu{\mid}}}

% change default figure size

\LetLtxMacro{\OldIncludegraphics}{\includegraphics}
\renewcommand{\includegraphics}[2][]{\OldIncludegraphics[width=3.25in,height=3.5in,keepaspectratio, #1]{#2}}

\captionsetup[figure]{margin=5pt,font=small,labelfont=bf,textfont=bf,justification=justified,}
\captionsetup[wrapfigure]{margin=5pt,font=small,labelfont=bf,justification=justified,singlelinecheck=off}
\captionsetup[table]{margin=5pt,font=small,labelfont=bf,textfont=bf,justification=justified,position=top}

\usepackage{lettrine} %The let­trine pack­age sup­ports var­i­ous dropped cap­i­tals styles
\usepackage{verbatim} %reimplements verbatim, block comments
\usepackage{multicol}
\usepackage{setspace} %double spacing for the manuscript
%\doublespacing

%nomenclature
\makenomenclature

\makeatletter
\@ifundefined{chapter}
  {\def\wilh@nomsection{section}}
  {\def\wilh@nomsection{chapter}}


\def\thenomenclature{%
 \begin{multicols}{2}[% 2 column layout
   \csname\wilh@nomsection\endcsname*{\nomname}
   \if@intoc\addcontentsline{toc}{\wilh@nomsection}{\nomname}\fi
   \nompreamble]
 \list{}{%
   \labelwidth\nom@tempdim
   \leftmargin\leftmargini
   \advance\leftmargin\leftmargini
   \itemsep\nomitemsep
   \let\makelabel\nomlabel}%
}
\def\endthenomenclature{%
 \endlist
 \end{multicols}
 \nompostamble}
\makeatother

%these seemed to be necessary with minted... not sure what they do..
\makeatletter
\color{black}
\let\default@color\current@color
\makeatother

%reference appendix
\crefname{appsec}{Appendix}{Appendices}
\begin{document}
\maketitle
\subimport{sections/}{abstract}
\newpage
%\tableofcontents
%\listoffigures
%\listoftables
\printnomenclature
\section{Introduction}
	\subimport{sections/}{introduction}
\section{Hyperloop System Model Overview}
	\subimport{sections/}{model_overview}
\section{Subsystem Analyses and Optimizations}
	\subimport{sections/}{subsystem_analyses_and_optimizations}
\section{Results}
	\subimport{sections/}{results}
\section{Conclusion}
	\subimport{sections/}{conclusion}
\newpage
\section{Appendix}
	\subimport{sections/}{appendix}

\bibliographystyle{unsrt} %sorted by appearence, consider using natbib?
\bibliography{heading}


\nomenclature{$D$}{Length (m)}
%\nomenclature{$P_{rad}$}{Radiated Power (W)} %does not appear
\nomenclature{$\nu$}{Kinematic Viscosity ($\frac{m^{2}}{s}$)} % previously upsilon
\nomenclature{$Pr$}{Prandtl Number}
\nomenclature{$h$}{Heat transfer coefficient ($\frac{W}{m^{2}K}$)}
\nomenclature{$k$}{Thermal Conductivity ($\frac{W}{mK}$)}
\nomenclature{$p$}{Pressure ($Pa$)}
%\nomenclature{$P_f$}{Final pumpdown pressure ($\frac{N}{m^{2}}$)}
\nomenclature{$\Pi$}{Pressure Ratio}
\nomenclature{$P$}{Power (W)}
%\nomenclature{$\dot{m}$}{Mass flow rate ($\frac{kg}{s}$)}
\nomenclature{$T$}{Temperature ($K$)}
\nomenclature{$C$}{Cost (USD)}
\nomenclature{$C_{p}$}{Heat capacity at constant pressure ($\frac{J}{kgK}$)}
\nomenclature{$C_{D}$}{Coefficient of drag}
\nomenclature{$Q$}{Heat energy ($J$)}
\nomenclature{$\gamma$}{Heat Capacity Ratio}
\nomenclature{$M$}{Mach Number}
\nomenclature{$\rho$}{Density ($\frac{kg}{m^3}$)}
\nomenclature{$A$}{Area ($m^{2}$)}
\nomenclature{$A^{*}$}{Throat Area ($m^{2}$)}
%\nomenclature{$P_{crit}$}{Critical pressure at which bucking occurs ($Pa$)}
\nomenclature{$E$}{Elastic modulus ($Pa$)}
\nomenclature{$\nu_{Poi}$}{Poisson Ratio} %was \nu
\nomenclature{$d$}{Thickness ($m$)}
\nomenclature{$d_{c}$}{Track strip spacing ($m$)}
\nomenclature{$r$}{Radius ($m$)}
\nomenclature{$\delta^{*}$}{Disp. Boundary layer thickness ($m$)}
%\nomenclature{$x$}{Distance in x direction ($m$)}
\nomenclature{$\varepsilon$}{Expansion ratio}
\nomenclature{$ib$}{Bond interest rate}
\nomenclature{$t$}{Time (variable)}
\nomenclature{$bm$}{Bond maturity (years)}
\nomenclature{$g$}{Gravitational acceleration ($\frac{m}{s^2}$)}
\nomenclature{$v$}{Velocity ($\frac{m}{s}$)}
\nomenclature{$x_{P}$}{Pump uptime per day}
%\nomenclature{$F_{x}$}{Magnetic Drag ($N$)} % repeast of Dmag
%\nomenclature{$D_{mag}$}{Magnetic Drag ($N$)} % repeat of Fx
\nomenclature{$F$}{Force ($N$)}
\nomenclature{$m$}{Mass ($kg$)}
\nomenclature{$W$}{Energy ($J$)}
\nomenclature{$w_{mag}$}{Magnet Width ($m$)}
\nomenclature{$\alpha$}{Weighting factor}
\nomenclature{$\lambda$}{Wavelength ($m$)}
%\nomenclature{$w$}{Width of the Magnetic Array ($m$)}
\nomenclature{$\beta$}{Magnetic Field Strength ($T$)}
\nomenclature{$R$}{Resistance ($\Omega$)}
\nomenclature{$L$}{Inductance ($H$)}
\nomenclature{$Re$}{Reynold's number}
\nomenclature{$\omega$}{Frequency ($Hz$)}
\nomenclature{$H$}{Height ($m$)} % previously h
%\nomenclature{$A_a$}{Area of Magnetic Array ($m^{2}$)}

\end{document}
