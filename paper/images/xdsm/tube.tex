\documentclass{article}
\usepackage{geometry}
\usepackage{amsfonts}
\usepackage{amsmath}
\usepackage{amssymb}
\usepackage{tikz}

% Define the set of tikz packages to be included in the architecture diagram document

\usetikzlibrary{arrows,chains,positioning,scopes,shapes.geometric,shapes.misc,shadows} 

% Set the border around all of the architecture diagrams to be tight to the diagrams themselves
% (i.e. no longer need to tinker with page size parameters)

\usepackage[active,tightpage]{preview}
\PreviewEnvironment{tikzpicture}
\setlength{\PreviewBorder}{5pt}


\begin{document}

% Define all the styles used to produce XDSMs for MDO

\tikzstyle{every node}=[font=\rmfamily]

% Component types
\tikzstyle{Optimization} = [rounded rectangle,draw,fill=blue!20,inner sep=6pt,minimum height=1cm,text badly centered]
\tikzstyle{LP_Optimization} = [rectangle,draw,fill=blue!20,inner sep=6pt,minimum height=1cm,text badly centered]
\tikzstyle{Analysis} = [rectangle,draw,fill=green!20,inner sep=6pt,minimum height=1cm,text badly centered]
\tikzstyle{AnalysisGroup} = [rounded rectangle,draw,fill=green!20,inner sep=6pt,minimum height=1cm,text badly centered]
\tikzstyle{ImplicitAnalysis} = [rectangle,draw,fill=red!20,inner sep=6pt,minimum height=1cm,text badly centered]
\tikzstyle{Function} = [rectangle,draw,fill=purple!20,inner sep=6pt,minimum height=1cm,text badly centered]
\tikzstyle{FunctionGroup} = [rounded rectangle,draw,fill=red!20,inner sep=6pt,minimum height=1cm,text badly centered]
\tikzstyle{MDA} = [rounded rectangle,draw,fill=orange!20,inner sep=6pt,minimum height=1cm,text badly centered]
\tikzstyle{Metamodel} = [rectangle,draw,fill=yellow!20,inner sep=6pt,minimum height=1cm,text badly centered]
\tikzstyle{DOE} = [rounded rectangle,draw,fill=yellow!20,inner sep=6pt,minimum height=1cm,text badly centered]
%\tikzstyle{OptFunction} = [rectangle,draw,fill=red!20,inner sep=6pt,minimum height=1cm,text badly centered]

%% A simple command to give the repeated structure look for components and data
\tikzstyle{stack} = [double copy shadow]

%% A simple command to fade components and data, e.g. demonstrating a sequence of steps in an animation
\tikzstyle{faded} = [draw=black!50,fill=white,text opacity=0.5]

%% Simple fading commands for the lines
\tikzstyle{fadeddata} = [color=black!20]
\tikzstyle{fadedprocess} = [color=black!50]

% **OLD** Component types for repeated structures (i.e. for parallel structures)
%\tikzstyle{Optimization_i} = [double copy shadow, Optimization]
%\tikzstyle{LP_Optimization_i} = [double copy shadow, LP_Optimization]
%\tikzstyle{Analysis_i} = [double copy shadow, Analysis]
%\tikzstyle{Function_i} = [double copy shadow, Function]
%\tikzstyle{MDA_i} = [double copy shadow, MDA]
%\tikzstyle{Metamodel_i} = [double copy shadow, Metamodel]
%\tikzstyle{DOE_i} = [double copy shadow, DOE]

% **OLD** Faded component types for, e.g. demonstrations of each step. We use these style definitions to "gray out" large parts of the diagram.
%\tikzstyle{Optimization_fade} = [Optimization,fill=blue!10,draw=black!30,text opacity=0.3]
%\tikzstyle{Analysis_fade} = [Analysis,fill=green!10,draw=black!30,text opacity=0.3]
%\tikzstyle{Function_fade} = [Function,fill=purple!10,draw=black!30,text opacity=0.3]
%\tikzstyle{MDA_fade} = [MDA,fill=orange!10,draw=black!30,text opacity=0.3]
%\tikzstyle{Metamodel_fade} = [Metamodel,fill=yellow!10,draw=black!30,text opacity=0.3]
%\tikzstyle{DOE_fade} = [DOE,fill=yellow!10,draw=black!30,text opacity=0.3]
%
%\tikzstyle{Optimization_i_fade} = [Optimization_i,fill=blue!10,draw=black!30,text opacity=0.3]
%\tikzstyle{Analysis_i_fade} = [Analysis_i,fill=green!10,draw=black!30,text opacity=0.3]
%\tikzstyle{Function_i_fade} = [Function_i,fill=purple!10,draw=black!30,text opacity=0.3]
%\tikzstyle{MDA_i_fade} = [MDA_i,fill=orange!10,draw=black!30,text opacity=0.3]
%\tikzstyle{Metamodel_i_fade} = [Metamodel_i,fill=yellow!10,draw=black!30,text opacity=0.3]
%\tikzstyle{DOE_i_fade} = [DOE_i,fill=yellow!10,draw=black!30,text opacity=0.3]

% Data types
\tikzstyle{DataInter} = [trapezium,trapezium left angle=75,trapezium right angle=105,draw,fill=black!10]
\tikzstyle{DataIO} = [trapezium,trapezium left angle=75,trapezium right angle=105,draw,fill=white]

% **OLD** Data types for repeated structures
%\tikzstyle{DataInter_i} = [double copy shadow, DataInter]
%\tikzstyle{DataIO_i} = [double copy shadow, DataIO]

% **OLD** Faded data types
%\tikzstyle{DataInter_fade} = [DataInter,draw=black!30,fill=white,text opacity=0.3]
%\tikzstyle{DataIO_fade} = [DataIO_i,draw=black!30,fill=white,text opacity=0.3]
%
%\tikzstyle{DataInter_i_fade} = [DataInter_i,draw=black!30,fill=white,text opacity=0.3]
%\tikzstyle{DataIO_i_fade} = [DataIO_i,draw=black!30,fill=white,text opacity=0.3]

% Edges
\tikzstyle{DataLine} = [color=black!40,line width=5pt]
\tikzstyle{ProcessHV} = [-,line width=1pt,to path={-| (\tikztotarget)}]
\tikzstyle{ProcessTip} = [-,line width=1pt]

% **OLD** Faded edges
%\tikzstyle{DataLine_fade} = [DataLine,color=black!10]
%\tikzstyle{ProcessHV_fade} = [ProcessHV,color=black!30]
%\tikzstyle{ProcessTip_fade} = [ProcessTip,color=black!30]

% Matrix options
\tikzstyle{MatrixSetup} = [row sep=3mm, column sep=2mm]

% Declare a background layer for showing node connections
\pgfdeclarelayer{data}
\pgfdeclarelayer{process}
\pgfsetlayers{data,process,main}

% A new command to split the component text over multiple lines
\newcommand{\TwolineComponent}[3]
{
    \begin{minipage}{#1}
    \begin{center}
        #2 \linebreak #3
    \end{center}
    \end{minipage}
}

\newcommand{\ThreelineComponent}[4]
{
    \begin{minipage}{#1}
    \begin{center}
        #2 \linebreak #3 \linebreak #4
    \end{center}
    \end{minipage}
}

% A new command to split the component text over multiple columns
\newcommand{\MultiColumnComponent}[5]
{
    \begin{minipage}{#1}
    \begin{center}
    #2 \linebreak #3
    \end{center}
    \begin{minipage}{0.49\textwidth}
    \begin{center}
    #4
    \end{center}
    \end{minipage}
    \begin{minipage}{0.49\textwidth}
    \begin{center}
    #5
    \end{center}
    \end{minipage}
    \end{minipage}
}


\begin{tikzpicture}

  \matrix[MatrixSetup]
  {
    %Row 1
    \node [Function] (SubTube) {\Large \TwolineComponent{6em}{Submerged}{Tube}}; &
    &
    &
    &
    &
    &
    \\
    &
    \node [AnalysisGroup] (Temp) {\Large \TwolineComponent{6em}{Tube}{Temperature}}; &
    \node [DataInter] (PropMech-Temp) {$T_\text{tube}$}; &
    \node [DataInter] (SSVacuum-Temp) {$T_\text{tube}$}; &
    &
    \node [DataInter] (Power-Temp) {$T_\text{tube}$}; &
    \\
    &
    &
    \node [Function] (PropMech) {\Large \TwolineComponent{6em}{Propulsion}{Mechanics}}; &
    &
    &
    \node [DataInter] (Power-PropMech) {$Pwr_\text{required}$}; &
    \\
    &
    &
    &
    \node [FunctionGroup] (SSVacuum) {\Large \TwolineComponent{6em}{Steady State}{Vacuum}}; &
    &
    &
    \\
    &
    &
    &
    &
    \node [Function] (Vacuum) {\Large Vacuum}; &
    \node [DataInter] (Power-Vacuum) {$E_\text{total}, Pwr_\text{total}$}; &
    \node [DataInter] (TubePylon-Vacuum) {$w_\text{total}$}; \\
    %Row 6
    &
    &
    &
    &
    &
    \node [Function] (Power) {\Large Tube Power}; &
    \\
    &
    &
    &
    &
    &
    &
    \node [Function] (TubePylon) {\Large \TwolineComponent{6em}{Tube}{and Pylon}}; \\
    %Row 8
  };

  \begin{pgfonlayer}{data}
    \path
    % Horizontal edges
    (Temp) edge [DataLine] (Power-Temp)
    (PropMech) edge [DataLine] (Power-PropMech)
    (Vacuum) edge [DataLine] (TubePylon-Vacuum)
    % Vertical edges
    (PropMech-Temp) edge [DataLine] (PropMech)
    (SSVacuum-Temp) edge [DataLine] (SSVacuum)
    (Power-Temp) edge [DataLine] (Power)
    (TubePylon-Vacuum) edge [DataLine] (TubePylon)
    ;
  \end{pgfonlayer}

\end{tikzpicture}

\end{document}
