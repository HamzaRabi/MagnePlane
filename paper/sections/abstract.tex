\begin{abstract}
	The Hyperloop concept is proposed as a faster, cheaper alternative to
	high-speed rail and traditional short-haul aircraft. It consists of a
	passenger pod traveling through a tube under light vacuum while being
	propelled and levitated by a combination of permanent and electro-magnets.
	The concept addresses NASA's research thrusts for growth in demand,
	sustainability, and technology convergence for high-speed transport.
	Hyperloop is a radical departure from other advanced aviation concepts,
	however it remains an aeronautics concept that tackles the same strategic
	goals of low-carbon propulsion and ultra-efficient vehicles.

	System feasibility was investigated by building a
	multidisciplinary vehicle sizing model that takes into account
	aerodynamic, thermodynamic, structures, electromagnetic, weight, and
	mission analyses. The sizing process emphasized the strong coupling
	between the two largest systems: the tube and the passenger pod. The model
	was then exercised to examine Hyperloop from a technical and cost perspective.
	The structural sizing analysis of the travel tube demonstrates potential for
	significant capital cost reductions by considering an underwater route.
	Examination of varying passenger capacity indicates that the system
	can be operated with a wide range of passenger loads without
	significant change in operating expenses. Lastly, a high-level sizing study
	simulated variations in tube area, pressure, pod speed, and passenger
	capacity showing that there is a tube pressure that minimizes operating
	energy usage. The value of this optimal tube pressure is highly sensitive
	to numerous design details. These combined estimates of energy consumption,
	passenger throughput, and mission analyses all support Hyperloop as a
	faster and cheaper alternative to short-haul flights.

	The tools and expertise used to quantify these results also
	demonstrate how traditional aerospace design methods can be leveraged to
	handle the complex and coupled design process. Much of the technology
	development required for the Hyperloop is shared with next-generation aircraft.
	Furthermore, the substantial public interest and active commercial
	development make it an ideal candidate as an aircraft technology driver
	and test bed.
\end{abstract}
