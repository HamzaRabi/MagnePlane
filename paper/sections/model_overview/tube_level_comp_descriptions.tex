	Once the Pod group determines the design configuration of the pod, the
	dimensions are fed into the Tube group. This group contains the subsystem
	analyses for the vacuum pumps, electromagnetic propulsion system, thermal
	management, and tube structure. Once the major aspects of the design of the
	tube are evaluated, results for both Pod and Tube design can be fed into
	Mission and Cost analyses to evaluate overall system performance
	requirements and their resulting costs.


\subsubsection{Vacuum}
	The vacuum subsystem is a group that evaluates two different
	operations: pump down and steady-state usage. The pump down module
	evaluates the number of vacuum pumps and the energy required to drop the pressure
	from ambient to the operating condition. This
	information is critical because the vacuum will need to be pumped down
	once the tube is constructed and in the event of an emergency pressurization.
	The energy required to draw down the tube pressure is given in the equation

	\begin{equation}
		\label{eq:vacuum}
		E_{tot} = pwr * \frac{vol_{tube}*speed_{pump}*log(P_{0}/P_{f})*2}{t_{pumpdown}}* x_{P} * 86400 \frac{sec}{day}
	\end{equation}
	The energy usage for this operation will be a critical element of
	operating cost. The pump down operation is only the first step in getting
	the travel tube down to operational pressure. Once that is done, we enter
	a steady-state phase. In theory, if the tube is perfectly air tight, there
	would be no need for the vacuum to run at all once the tube is pumped
	down. However, air is likely to leak into the system throughout normal
	operation, particularly during loading of pods into the tube.
	Consequently, vacuum pumps will need to be used throughout operation to
	maintain operating pressure. The effect of leakage rate on energy
	consumption is critical and will be discussed in more depth later. For the
	purpose of this analysis, the steady-state vacuum pump is modeled as a
	compressor using the same type of thermodynamic model employed in the pod
	cycle analysis. The pressure ratio
	of the vacuum will be equal to the ratio of ambient pressure to desired
	tube pressure and the mass flow through the vacuum at steady state will be
	equal to the leakage rate.
\subsubsection{Thermal Management}
	A basic heat balance is performed to determine the approximate tube
	temperature. The quasi-steady analysis ignores heat transfer time lags and
	diurnal cycles. Heat is assumed to transfer and equilibrate rapidly
	between the pod compressor system, the rarified atmosphere inside the
	tube, the tube walls, and the outside environment. This analysis is
	based off previous work by Chin et al.\cite{Chin} and concludes manageable heat
	loads for air-cabin cooling. A conservative equilibrium tube temperature,
	with the tube in direct sunlight, on a 90 degree Fahrenheit hot day and
	during maximum pod operation, is on the order of 110 degrees Fahrenheit.
	For an underground or underwater trajectory, these estimates would drop even lower.
\subsubsection{Electromagnetic Propulsion}
	A series of linear synchronous motors (LSMs) is proposed to accelerate the
	pod from rest to top speed and maintain top speed with periodic boosts.
	While the specifications of the LSM system design are beyond the scope of
	this analysis, the amount of energy and power required of an LSM can be
	determined using the simple mechanics relationship,
	\begin{equation}
		\label{eq:sum_of_forces}
		F_{net} = F_{LSM} + F_{thrust} - \frac{1}{2}C_{D}\rho V^{2}S - D_{mag}
	\end{equation}
	in which $F_{LSM}$ is the force required of the LSM system. This equation
	is integrated to determine the power and energy requirements for both
	startup and coasting booster sections. For the purpose of this analysis,
	the efficiency of the LSM is assumed to be 0.8 based on rough approximations \cite{LSM}.
\subsubsection{Structure}
	The Structural group determines the structural design of the tube
	for two phases of travel: travel over land and under water. When traveling
	overland, the tube is assumed to be supported by pylons above the terrain
	at a given height. For travel under water, the tube is notionally
	supported at a certain depth below sea level. The structural analysis in
	each phase is under constrained and allows for several free choices to be
	made by the user, which could have a significant impact on design
	configuration and material cost. To handle this challenge, the structural
	design of the tube at each phase was optimized in order to determine
	the configuration that minimizes cost. The optimization methods will be
	discussed later in further detail.

