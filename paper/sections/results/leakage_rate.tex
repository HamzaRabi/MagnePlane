A major goal of this model is to study the affect that leakage has on optimal tunnel configuration. The Hyperloop infrastructure will be designed and manufactured with the goal of making the tube air tight; however, a perfectly air tight tube is likely not possible. Slight leakage flow rates are to be expected due to diffusion, desorption, permeation, leaks through micro cracks, and leaks in mechanical mechanisms. Furthermore, air will likely need to be introduced to the system to allow passengers to board. The leakage rate due to diffusion and permeation is proportional to the tube surface area while the air introduced into the system during passenger boarding is proportional to the size of the boarding area and the frequency at which pods are boarded by passengers. Thus, it is reasonable to assume that leakage rate is proportional to pod size and the frequency at which pods leave the station. While the exact leakage rate into the system is difficult to quantify, accounting for leakage provides valuable insight to performance of the system as a whole. 
The previous trade study showed that, for a given leakage rate, an operating pressure exists at which steady state energy consumption is minimal. This is because an increase in pressure decreases the energy required by the vacuum pumps to maintain tube pressure while increasing the energy required by the compressor system onboard the pod to compress higher mass flows. The pressure at which this minimum occurs is a function of leakage rate. To evaluate the sensitivity of optimum pressure to leakage rate, the previous study will be repeated for a range of leakage rates. In each study, the pressure at which minimum energy consumption occurs will be determined using the ScipyOptimizer. The minimum value of total energy cost and the corresponding tube pressure will be recorded and plotted to evaluate the effects that leakage rate has on the system design variables. For this analysis, the values of other relevant design parameters are listed in $Insert Table number$.
\begin{figure}
	\centering
	\includegraphics{../../press_vs_leakage_rate.png}
	\caption{Pressure at Minimum Energy Consumption vs. Leakage Rate}
	\label{fig:pres_vs_leakage_rate}
\end{figure}
\begin{figure}
	\centering
	\includegraphics{../../images/pres_vs_leakage_rate.png}
	\caption{Minimum Energy Cost vs. Leakage Rate}
	\label{fig:pres_vs_leakage_rate}
\end{figure}
\Cref{fig:pres_vs_leakage_rate} shows the pressure at minimum energy consumption vs. the leakage rate of the tube. The tube pressure that optimizes cost increases as leakage rate increase. This trend is reasonable because increasing leakage rate increases the power required for the vacuum pumps to maintain the tube pressure, which is offset by increasing the pressure that the vacuum is required to maintain. This relationship is critical because it reveals a coupling between tunnel leakage and energy consumption that system designers must consider. As \cref{fig:pres_vs_leakage_rate} reveals, changes in leakage rate can have a significant effect on energy consumption and energy cost. Increasing leakage rate causes minimum energy consumption to increase significantly, with the cost increase being even more substantial when operating at a suboptimal pressure. If the designer wants to optimize the system by minimizing the energy consumption, then more accurate modeling or empirical studies will be necessary to determine operating pressure. Furthermore, the assumption made in this study that leakage rate is constant is likely not indicative of a real system. As pod frequency changes, the leakage rate, and therefore the optimal pressure of the tunnel, is also likely to change. It may be possible for system designers to account for variable tube pressure when sizing the battery, compressor motor, and tube diameter in order to give the operator flexibility to change the tube pressure with pod frequency. This would allow the Hyperloop system to adapt to changing operating conditions to more closely track optimal design configurations in real time. Further research with higher fidelity modeling is necessary to further characterize the benefits of variable tunnel pressure as it relates to pod frequency and leakage.
