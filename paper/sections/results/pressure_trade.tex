One of the most critical design variables of the Hyperloop is the tube pressure at which it operates. A lower pressure increases the power required of the vacuum system in order to pump the tube down and maintain tube pressure for a given leakage rate. However, it is known that the pressure at which the tube operates will increase the density of the air and thus increase the power requirements of both the electromagnetic propulsion system and the onboard compressor. As compressor power demands change, the length and mass of the motor and battery will change the pod geometry. Tube area is coupled with pod length, as was shown in the boundary layer analysis, while the power required of the electromagnetic propulsion system is coupled with pod mass. The boundary layer that was described previously is used for this analysis. Tube pressure is perhaps the single most important design variable to understand because it is coupled with so many different aspects of system construction and performance. Trade studies using this fully comprehensive system model can provide valuable insight into the effects that variations in pressure have in every aspect of design and performance at the system level.
Previous research has suggested that the ideal operating pressure should be on the order of 100 Pa \cite{Musk, Chin}. However, such a low pressure could be costly and difficult to implement across such a larger stretch of tube, especially if there is significant leakage. It is likely that slight increases in tube pressure could result in significant decreases in energy consumption that could make up for increased drag penalty. To evaluate this, the full system model will be run for a range of pressures. Tube area, compressor power, steady state vacuum power, and total yearly energy consumption are recorded at each pressure. Other relevant design variables in this analysis are listed in the \Cref{tbl:FIX_THIS}.
\begin{table}
	\centering
	\caption{FIX THIS}
	\label{tbl:FIX_THIS}
	% \begin{tabular}
	% \end{tabular}
\end{table}
\begin{figure}
	\centering
	\includegraphics{../../tube_area_vs_tube_press}
	\caption{Tube Area vs. Tube Pressure}
	\label{fig:tube_area_vs_tube_press}
\end{figure}
\Cref{fig:tube_area_vs_tube_press} shows tube area as a function of tube pressure. As pressure increases, tube area decreases until leveling off around 3500 Pa. This relationship is due to the effect that pressure has on boundary layer thickness. Increasing the pressure increases the Reynolds number per unit length, which decreases boundary layer thickness. As boundary layer thickness is reduced the effective bypass area is increased which allows the tube area to be reduced for the same Mach number. However, compressor power increases linearly with pressure, which results in an increase pod length to hold a larger motor and battery. As was shown previously, increasing length results in increased boundary layer growth and causes tube size to grow. This trade results in two pressure regime, which are illustrated in \cref{fig:tube_area_vs_tube_press}. At lower pressures, marginal increases in pressure result in decreases in tube area because the increase in Reynolds number per unit length dominates the length increases necessitated by higher compressor power demands. Meanwhile, at higher pressures, increases in pod length begin to dominate boundary layer sizing and the marginal effect on tube area is tempered. 
\begin{figure}
	\centering
	\includegraphics{../../images/power_demands_vs_tube_press.png}
	\caption{Power Demands and Energy Consumption vs. Tube Pressure}
	\label{fig:pow_demands_vs_tube_press}
\end{figure}
\Cref{pow_demands_vs_tube_press} shows how the power and energy consumption change with tube pressure. As expected, high tube pressures require lower power for a given leakage rate while requiring a higher power output from the onboard compressor for a given compressor ratio. Thus, energy cost increases for very low pressures due to the vacuum system and increases for very high pressures due to increased power demand from the compressor. The second plot in \cref{pow_demands_vs_tube_press} shows that this relationship produces a pressure at which energy cost is minimal. In this case, the energy consumption is minimal at about 100 Pa. Energy consumption increases fairly rapidly as pressure is increased beyond this minimum value. It is important to note that this exact value is dependent on the leakage rate, compressor pressure ratio, and whether or not regenerative breaking is used to recover battery energy (no regenerative breaking is assumed in this analysis). However, this relationship is crucial because it means that there does exist a pressure that optimizes energy cost and that energy cost can increase rapidly if deviations from this optimum point exist. A higher fidelity model can be used to determine exactly what value of pressure optimizes energy consumption for a given configuration. A more detailed examination of the effects that leakage has on optimum pressure will be conducted next. 
