Previous research analyzing the performance of the Hyperloop system suggested that the pod would likely be able to carry about 28 passengers \cite{Musk}. To meet the market demand, the frequency at which pods departed could be increased or decreased as necessary. Increasing pod frequency too high could be problematic as this could require a large number of pods to be maintained at the end points and may not provide enough time for passengers to board comfortably. Thus, it is of particular interest to examine how the performance of the system is affected over a range of pod capacities. Increasing the capacity would allow the user to have fewer pods taking off at a lower frequency in order to carry a given number of passengers per year. The overall benefits of changes in pod capacity can be more accurately determined by analyzing the sensitivity of energy consumption and operating cost to pod capacity.
In this analysis, the number of passengers is varied over an order of magnitude. At each quantity, the energy cost and estimated ticket cost are recorded. Other relevant design variables are given in $Insert Table Number$.
\begin{figure}
	\centering
	\includegraphics{../../images/energy_cost_vs_passengers_per_pod.png}
	\caption{Yearly Energy Cost vs. Passengers per Pod}
	\label{fig:energy_cost_vs_passengers}
\end{figure}
\Cref{energy_cost_vs_passengers} shows the relationship between yearly energy consumption and the number of passengers per pod produced buy the system model. It is shown that, for the given operating condition, an order of magnitude increase in pod capacity only results in a 15\% increase in yearly energy consumption. This, in conjunction with the previously discussed structural analysis, indicates that the cost associated with changing pod capacity is small. This relationship is significant because it means that the Hyperloop operator can specifically set the pod capacity to whatever value is necessary to meet a particular market demand without costly changes in performance or design. Furthermore, this makes it possible for future researchers to consider making Hyperloop pods modular. It is possible that, instead of having one large pod carrying a fixed number of passengers, the operator could have multiple pods that carry a small number of passengers each link together until the capacity of each individual flight reaches the value required by the market. This would allow the Hyperloop to handle high densities of passengers during peak travel times without having to increase pod frequency to prohibitive levels. Then, during lighter travel times, the operator could link fewer pods together to reduce the gross weight of each flight in order to reduce unnecessary energy consumption. The performance of the Hyperloop system scales favorably with pod capacity, which could potentially allow the system to be optimized to meet the demands of the market with little cost to operator.
