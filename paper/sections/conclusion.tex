Using openMDAO and pyCycle2, a fully open source model of the Hyperloop concept was created to evaluate engineering trade studies at the top level of the system. The analyses presented illustrate the sensitivity of the overall system design and performance to several key design variables. First, it is shown that the coupling between tube pressure and the energy consumption of different system components result in the existence of an optimal tube pressure at which total energy consumption is minimal. This optimal pressure and minimum energy consumption are shown to be coupled to the tunnel leakage rate. Second, this paper supports previous research that traveling above Mach .8 is likely not practical because it requires a large tube size. The coupling between tube size and boundary layer growth is analyzed to show that the required size of the tube is very sensitive to boundary layer growth. As a result, further research on the modeling and implementation active flow control is recommended due to its potential to significantly reduce required tube size. Finally, the system is shown to scale favorably with pod capacity. This allows each pod to carry more passengers than was previously proposed, meaning the pod capacity can be optimized to meet market demand without prohibitive costs to the operator. Although this work is not high fidelity, the trends and trade studies identified provide valuable insight into the physics behind the Hyperloop concept and how these physical relationships can inform future design efforts. The open source, modular nature of this system model will allow future researchers to modify, adapt, and improve the model to include more specific subsystems and higher fidelity modeling as needed. The modeling platform is intended to serve as a publicly accessible baseline that is easy to expand and delve deeper into this unique multidisciplinary system.