Using OpenMDAO, a fully open-source model of the Hyperloop was created to
evaluate system and subsystem level engineering and cost trade studies with
the overall goal of better characterizing the concept. The key subsystem
models included structural analyses of the travel tube, RANS CFD based drag
model for the passenger pod, electromagnetic analysis of the levitation
system, an empirical battery model, thermodynamic analysis of the vacuum
pumping system, and a numerical integration based mission analysis. The trade
studies provides several key conclusions about the sensitivity of Hyperloop to
key design parameters. A subsystem analysis of the structural requirements
for the tube showed that underwater routes could have a significantly lower
materials cost due to a more favorable loading conditions generated by
buoyancy.

A full model system study was performed to analyze the effect of the vacuum
system, which maintains the tube operating pressure, on net energy usage as a
function of the leakage rate of air into the tube. The results show that the
energy usage from this system is of the same order of magnitude as the energy
required to propel the passenger pod. Furthermore, the two systems have
opposite sensitivities with respect to tube pressure. As the tube pressure is
lowered, the energy needed to propel the pod drops but the energy required to
maintain the vacuum goes up. The results show that there is
an optimal tube operating pressure that is heavily dependent on the leakage
rate. The higher the leakage rate is, the higher the optimal tube operating
pressure becomes.

This work also extends previous research that shows traveling at speeds above
Mach .8 is likely not practical. The tube size invariably becomes too large.
Prior work showed that the was a coupling between tube size and pod travel
speed. We refine that analysis to include the effects of boundary layer growth
along the passenger pod. The data shows that boundary layer growth amplifies
the coupling between tube size and travel speed and hence is an important
consideration in Hyperloop design. As a result, further research on the
modeling and implementation active flow control is recommended due to its
potential to significantly reduce required tube size.

Finally, net energy usage is found to be relatively insensitive to pod length.
Therefore, the system would scale favorably to much higher passenger
capacities than originally proposed.
This also gives the operator freedom to vary capacity by lengthening or
shortening pods, meaning travel capacity can be optimized to meet market demand
without prohibitive costs to the operator.

Although the models presented in this paper are not of high
fidelity, the trends and trade studies identified provide valuable insight
into the engineering behind the Hyperloop concept and how these physical
relationships can inform future design efforts. The open source, modular
nature of this system model will allow future researchers to modify, adapt,
and improve the model to include more specific subsystems and higher fidelity
modeling as needed. The modeling platform is intended to serve as a publicly
accessible baseline that is easy to expand and delve deeper into this unique
multidisciplinary system.