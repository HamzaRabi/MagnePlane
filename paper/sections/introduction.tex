\paragraph{Motivation}

	Currently, flying is the fastest and cheapest way for commercial
	passengers to travel long distances. However, for shorter routes of less
	than 500 miles the efficiency of air transport erodes quickly. The
	inherent inefficiency of air travel (e.g. arriving at the airport early,
	and taxi time) can take up more time than the actual flight. Additionally,
	many commercial aircraft designed to accommodate much longer missions of
	over 3000 miles and hence are burn more fuel on shorter missions than a
	more specialized design would. To Hyperloop is a new commercial
	transportation concept designed to offer both a shorter travel time and
	lower ticket cost for short-haul routes. In order to achieve a
	simultaneous travel time and cost reductions, Hyperloop merges a
	combination of aerospace and ground transportation technologies to achieve
	fundamentally new capability.

	The most important feature of Hyperloop is that it has a sealed passenger
	pod traveling through a partially evacuated tube on a low-friction
	levitation system, at ground level. Staying at ground level means the
	passenger pod will not expend energy climbing and descending thus saving
	significant energy. The partial vacuum in the travel tube lowers the pod
	drag and enables it to travel at speeds around Mach .8, which is how it
	keeps travel times so short. Traveling in a sealed tube also enables
	Hyperloop to operate independent of weather conditions, which is another
	advantage over air transport. There are a number of proposed methods for
	handling vehicle levitation and propulsion. The original Hyperloop Alpha
	proposal suggested the use of linear air-bearings. Subsequent
	efforts have focused more on magnetic levitation, or mag-lev, systems
	similar to those developed for high-speed-rail (HSR) applications. Passive
	mag-lev systems offer only levitation and rely an externally mounted
	linear accelerators for propulsion. Active mag-lev systems can provide
	both levitation and propulsion, though some designs still incorporate
	external linear accelerators.

	Since Hyperloop travels in a partial vacuum it must still contend with
	aerodynamic drag, though the magnitude of that drag is dramatically
	reduced. The challenge is that as the passenger pod travels down the tube
	at a high rate of speed it has a tendency to drag some of the air along
	with it which causes a pressure imbalance to build up around the pod an
	increase drag over time. It is possible to design the Hyperloop tube to
	avoid this drag build up, but the required tube diameter to avoid it will
	grow as a the pod travel speed increases. At speeds around Mach .8, the
	design cruising speed, the required tube size would be on the order of 4
	meters in diameter. Travel tubes of this size will contribute
	substantially to the capital costs for constructing a Hyperloop system.
	In order to reduce the tube size somewhat, Hyperloop uses an electrically
	driven compressor in the front of the vehicle to preventing the build up
	of pressure in front of the vehicle. Existing aerospace research for
	electric aircraft propulsion can be leveraged to provide the necessary
	technology for this compressor.

	The Hyperloop requires a fusion of technologies from the aerospace and HSR
	applications, though its transonic operation, air-breathing flow-path, and
	aerodynamic driven design give it qualities more similar to a plane rather
	than a train. The concept deviates from existing high-speed rail by using
	the partially evacuated travel tube to enable high speed operation. This
	change, fundamentally alters the design considerations necessary for
	Hyperloop and requires the consideration of many aerodynamic problems that
	are common to aircraft, but not ever seen in train design. Another key
	effect of the travel tube is that Hyperloop does not need to run on rails
	mounted on the ground. The travel tube can be either above ground on
	concrete columns, underground in subterranean tunnels, or in underwater
	tunnels while maintaining a relatively straight trajectory. While it would
	be possible for trains to also run in similar situations, Hyperloop
	passenger pods will be significantly lighter than train cars and hence it
	is much more feasible to have the tube mounted away from the ground. 

	The design mission for Hyperloop is between 250 and 500 nautical miles,
	which accounts for 57\% of commercial aircraft fleet operations in 2012.
	NASA's Strategic Implementation Plan stresses that global operations must
	keep pace with both an overall growing transportation market and
	simultaneous growth in market share dedicated to air transport transport.
	By 2050, 41\% of world traffic (and 71\% of North America) market share is
	projected to be air transport transport.\cite{Schafer} This growth will be
	limited the existing air-transportation infrastructure and will require
	dramatic improvements to air-traffic control capabilities to support
	increased numbers of flights in increasingly more congested air-space.

	The Hyperloop offers a compelling opportunity to offset this congestion by
	offering a faster and lower cost transportation option for a large portion
	of short-haul aviation routes. While Hyperloop does require new
	infrastructure to be built, there are numerous high-population adjacent
	cities exist that have sufficient commercial travel volume to warrant the
	construction costs. Aeronautics market research shows that demand for this
	travel segment is the most sensitive to technology improvements and ticket
	price.\cite{Baik} Hyperloop offers a significantly lower travel times and
	reduced cost, highlighting the high potential for its impact in the short-
	haul market. It has the potential to offer a compelling price point that
	would even be competitive considering automobile vehicle miles are
	projected to increase by 7 billion simply due to airline fare increases by
	10\% in 2015.

	Beyond the economic benefits of Hyperloop, it also offers a significant
	reduction in carbon emissions. Planes produce large quantities of carbon
	emissions, exhausted into the upper atmosphere resulting in
	environmentally damaging travel. The fully-electric Hyperloop is designed
	for highly trafficked corridors, where its high upfront cost can be
	offset by cheaper, faster, and cleaner commercial travel on large scales
	and the long travel tubes could be covered with solar panels to generate
	the needed power which lowers the net carbon footprint. The Hyperloop is
	intended to alleviate billions of commuter car passenger miles, as well as
	free up airspace, reducing congestion and travel times for flights that
	are well suited for the national airspace system.\\

\paragraph{History}

	Aerospace engineers have promoted tube transport for over a century.
	As early as 1972, a study conducted by the RAND corporation concluded that high-speed 
	`tubecraft' was technologically feasible, with political pressure being the greatest
	obstacle to creation.\cite{RAND} Many concepts later developed by
	the government, industry, and academia incorporated technologies from which
	Hyperloop can draw technical precedents.
	These include, but are not limited, to the Magneplane, Maglifter, and numerous
	other magnetic propulsion concepts. In 2013, Elon Musk, CEO of Space Exploration
	Technologies (SpaceX) and Tesla Motors revived the concept with the publication
	of his open-source paper, Hyperloop Alpha.\cite{Musk}
	Unlike previous waves of interest, this popularized design has spurred widespread international
	development efforts amongst leading universities, private companies with over
	\$100M in venture capitalist backing, and smaller research efforts at NASA and the
	United States Department of Transportation. \cite{Chin}

\paragraph{Focused Research Questions}

	The Hyperloop has largely evolved as an achievable concept over the past several years.
	In addition to Musk's paper, several key research questions were identified in a study conducted
	by the Department of Transportation (DOT) to quantify the potential value of the
	Hyperloop vehicle. \cite{Volpe} This paper works towards answering a few of
	those key drivers from both an engineering and business perspective.

	As highlighted by the DOT, the Hyperloop must be sufficiently lightweight
	that fixed capital costs of construction, when compared to conventional
	elevated HSR or Maglev systems, would be greatly decreased.
	Although true construction costs cannot be estimated through an engineering
	model alone, this paper seeks to minimize system size, energy, and material
	quantity as a proxy to cost. This model provides a sensitivity
	analysis of structural requirements as a function of pod mass to partially
	answer this research question. Rather than focus on weight of the system explicitly,
	many of the trends discussed are explored in relation to tube internal
	cross sectional area. Tube weight relates directly to this area term, and
	area is the common design variable linking many of the top-level
	analyses. As depicted in previous work,
	\cite{Chin} when abstracting the concept to three top level
	sizing variables (vehicle frontal area, speed, and tube cross-sectional area) only two are needed
	to find the third.

	Moreover, the potential ability to operate underwater is a key
	distinguishing feature of the Hyperloop versus conventional HSR and Maglev
	trains. It is possible for the buoyant force of the submerged tube to
	counter the systems weight, meaning weight is not a principle factor in
	this context. Given these considerations, the key research question can be
	restated as: ``Is the Hyperloop sufficiently compact so that there would
	be construction savings compared to conventional HSR or Maglev systems?''
	Although you could put a conventional train in an underwater tube, these
	systems would not be capable of traveling nearly as fast as the Hyperloop
	in a volume constrained environment. Therefore subsequent results are
	presented in relation to the tube area.

	Another research query, highlighted by the DOT questions, the ability to
	expand the throughput of the system beyond the 28 passengers per pod
	originally proposed. This scaling capacity is addressed
	via analysis of pod frequency combined with design sensitivity to pod length.
	The relationship between pod mass and number of passengers per pod is also
	considered in subsequent structural analyses.

	The DOT also highlights the need to determine maximum payload capacities
	of the pods. Adding minimum required weights as model constraints is beyond the scope of
	this paper, but could be addressed using the engineering methods outlined.
	Answering this question and better defining component weight breakdowns
	is recommended as a next step for future work.
	This paper also does not address land acquisition costs or throughput
	constraints in today's existing transportation infrastructures.
	However, the cost modules in this model could consider these constraints if
	reliable data becomes available in the future.
	Capacity constrained water ports and airports are critical drivers referenced
	both in the DOT study and the NASA Aeronautics Strategic
	Implementation Plan. The prospect of high-speed underwater transportation
	completely bypasses certain land acquisition challenges and provides a novel solution
	to eliminating bottlenecks in multimodal travel integration. The analyses performed in this paper
	are intended to optimize performance based on throughput and cruise conditions.
	Local transit presents different challenges and objectives than those seen in the longer distances
	considered in this study. This system model, however, could be easily adapted to perform
	analysis and optimization of local transit operation in the future.\\

\paragraph{Paper Scope}
	This paper introduces the first open-source system model to examine the
	complete energy consumption and converged vehicle model for the Hyperloop concept.
	The full-system engineering model is built in Python to analyze the couplings
	between various subsystems in the model, namely between the tube, passenger pod, and
	levitation systems. Using this model, trade studies were conducted to analyze the technological
	and economic feasibilities of the system.
	At the end, findings are presented regarding design aspects that simultaneously
	met all system constraints across each sub-discipline.

	The paper expands on previous works \cite{Chin}
	and is built using OpenMDAO, an open-source multidisciplinary analysis and
	optimization framework developed at the NASA Glenn Research Center.
	This work focuses on balancing the thermodynamic, aerodynamic, structural,
	weight, and power considerations of the system, with additional calculations for cost
	and mission design incorporated in the final system. Because each sub-system
	is designed independently of the whole, this model operates in a modular fashion, allowing
	for each subsystem the flexibility of being replaced with higher fidelity modules in the future.
	The model is not an attempt to obtain detail design conclusions for the Hyperloop
	-- rather, it is intended to perform sensitivity analyses
	and trade studies appropriate for evaluating the merit of the concept.

