\paragraph{Motivation}

	Currently, air travel is the most widespread method for high-speed travel over long distances.
	To reduce operating costs, commercial airliners often focus their efforts on optimizing aircraft for
	cruise conditions since this is the largest segment of travel over long distances.
	However, when traveling shorter distances (under 500 miles),
	vehicles do not spend a majority of their travel time in cruise.
	Instead, the time spent taxiing, taking off, climbing, descending, and landing
	takes up a larger portion of the total travel time. In general, aircraft are not optimized
	for these phases of travel, and the high fixed overhead in upfront travel time severely reduces the benefits of short-haul flights. The result is a mode of travel that is
	ill-suited both efficiency wise and environmentally for short missions.

	The Hyperloop is a high speed vehicle concept designed to meet the
	challenges facing the short-haul aviation industry.
	The concept alters the mission profile of traditional aviation by bringing
	the entire trajectory down to the ground level.
	As a result, a Hyperloop vehicle will not have to expend energy climbing and descending from
	a cruising altitude and can instead spend more time operating at cruise conditions.
	Drag is reduced by flying through a low-pressure tube
	and conventional wings are replaced by alternative types of lifting bodies
	that are more compact and have lower induced drag than conventional aerodynamic surfaces.
	Traveling in a tube allows the Hyperloop to leverage newer, cleaner propulsion technologies.
	An electromagnetic propulsion system can be located outside of the vehicle,
	which reduces the weight and mechanical complexity of the Hyperloop vehicle compared to a
	traditional aircraft with the same passenger throughput.
	The Hyperloop vehicle can also include a fully electric compressor in the front of the vehicle to
	further reduce drag by preventing the build up of pressure in front of the vehicle.
	The compressor can be fully electric, allowing the Hyperloop to take full
	benefit of recent technological developments in
	high-density electrical power storage and fully electric turbomachinery.
	The result is a technologically advanced vehicle that spends more time operating under optimal conditions
	all while releasing zero carbon emissions into the upper atmosphere.

	The design of the Hyperloop has continuously evolved, with the latest derivative
	generating lift using a specialized magnetic levitation (MagLev) system.
	In the same way an airplane is differentiated from an airship or a hydroplane is differentiated from
	a boat, the Hyperloop is a plane that generates lift dynamically
	by moving over a medium at high-speeds
	rather than depending on buoyancy or static phenomena.
	Its transonic operation, air-breathing flow-path and aerodynamic
	driven design give it qualities more similar to a plane rather than a train.
	The concept deviates from existing high-speed rail designs by enclosing the passenger pod in a tube under a partial vacuum
	propelled by a set of linear electromagnetic accelerators
	mounted to the tube. The entire system is held either above ground on concrete
	columns, underground in subterranean tunnels, or in underwater tunnels while maintaining a relatively straight trajectory.
	The Hyperloop flies low enough that propulsion can be offloaded to ground systems,
	intended to rival efficiencies previously limited to terrestrial vehicles.
	By sacrificing an airplane's traditional mission flexibility,
	Hyperloop can be optimized for passenger throughput, door-to-door travel time,
	and energy efficiency.
	This unique combinations of potential levitation and propulsion technologies,
	differentiate it from previous tube trains and maglev systems.

	The concept is notionally sized for distances between 250-500 nautical miles,
	which accounts for 57\% of commercial aircraft fleet operations in 2012. NASA's
	Strategic Implementation Plan stresses that global operations must keep pace with both an overall growing
	transportation market and simultaneous growth in market share dedicated to high-speed transport.
	By 2050, 41\% of world traffic market share is projected to be high-speed transport. \cite{Schafer}

	The Hyperloop offers a compelling opportunity to offset this congestion
	by occupying a highly desired gap in travel distances.
	Numerous high-population adjacent cities exist that are currently
	too far to conveniently travel by car and too short to efficiently travel by plane.
	Aeronautics market research shows that demand for this travel segment is
	the most sensitive to technology improvements and ticket price.\cite{Baik}
	The departure from typical aircraft constraints is intended to provide a
	compelling price point considering automobile vehicle miles are
	projected to increase by 7 billion simply due to airline
	fare increases by 10\% in 2015.

	Furthermore, planes produce large quantities of carbon emissions,
	exhausted into the upper atmosphere resulting in environmentally damaging travel.
	The fully-electric Hyperloop is designed for highly trafficked corridors,
	where it's high upfront cost can be offset by cheaper, faster, and cleaner
	commercial travel on large scales.
	The Hyperloop is intended to alleviate billions of commuter car passenger miles,
	as well as free up airspace, reducing congestion and travel times for
	flights that are well suited for the national airspace system.\\

\paragraph{History}

	Aerospace engineers have promoted tube transport for over a century.
	As early as 1972, a study conducted by the RAND corporation concluded that high-
	speed `tubecraft' was technologically feasible, with political pressure being the greatest
	obstacle to creation.\cite{RAND} Many concepts later developed by
	the government, industry, and academia incorporated technologies from which
	Hyperloop can draw technical precedents.
	These include but are not limited to the Magneplane, Maglifter, and numerous
	other magnetic propulsion concepts. In 2013, Elon Musk, CEO of Space Exploration
	Technologies (SpaceX) and Tesla Motors revived the concept with the publication
	of his open-source paper, Hyperloop Alpha.\cite{Musk}
	Unlike previous waves of interest, this popularized design has spurred widespread international
	development efforts amongst leading universities, private companies with over
	\$100M in venture capitalist backing, and smaller research efforts at NASA and the
	United States Department of Transportation. \cite{Chin}

\paragraph{Focused Research Questions}

	The Hyperloop has largely evolved as an achievable concept over the past several years.
	In addition to Musk's paper, several key research questions were identified in a study conducted
	by the Department of Transportation (DOT) to quantify the potential value of the
	Hyperloop vehicle. \cite{Volpe} This paper works towards answering a few of
	those key drivers from both an engineering and business perspective.

	As highlighted by the DOT, the Hyperloop must be sufficiently lightweight
	that fixed capital costs of construction when compared to conventional
	elevated HSR or Maglev systems would be greatly decreased.
	Although true construction costs cannot be estimated through an engineering
	model alone, this paper seeks to minimize system size, energy, and material
	quantity as a proxy to cost. This model provides a sensitivity
	analysis of structural requirements as a function of pod mass to partially
	answer this research question. Rather than focus on weight of the system explicitly,
	many of the trends discussed are explored in relation to tube internal
	cross sectional area. Tube weight relates directly to this area term, and
	area is the common design variable linking many of the other top-level
	design variables. As depicted in previous work,
	\cite{Chin} when abstracting the concept to three top level system level
	sizing variables;(vehicle area, speed, and tube area) only two are needed
	to find the third.

	Moreover, the potential ability to operate underwater is a key distinguishing feature
	of the Hyperloop over conventional HSR and Maglev trains. It is possible for the buoyant force
	of the submerged tube to counter the systems weight, meaning weight is not
	a principle factor in this context.
	Given these considerations, the key research
	question can be restated as: ``Is the Hyperloop sufficiently compact that
	there would be construction savings compared to conventional
	HSR or Maglev systems?'' Although you could put a conventional train in an
	underwater tube, these systems would not be capable of traveling nearly as
	fast as the Hyperloop in a volume constrained environment.
	Therefore subsequent results are presented in relation to the tube area.

	Another research question highlighted by the DOT questions the ability to
	expand the throughput of the system beyond the 28 passengers per pod
	originally proposed. This scaling capacity is addressed
	via analysis of pod frequency combined with design sensitivity to pod length.
	The relationship between pod mass and number of passengers per pod is also
	considered in subsequent structural analyses.

	The DOT also highlights the need to determine maximum payload capacities
	of the pods. Adding minimum required weights as model constraints is beyond the scope of
	this paper, but could be addressed using the engineering methods outlined.
	Answering this question, and better defining component weight breakdowns
	is recommended as a next step for future work.
	This paper also does not address land acquisition costs or throughput
	constraints in today's existing transportation infrastructures.
	However, the cost modules in this model could consider these constraints if
	reliable data becomes available in the future.
	Capacity constrained water ports and airports are critical drivers referenced
	both in the DOT study and the NASA Aeronautics Strategic
	Implementation Plan. The prospect of high-speed underwater transportation
	completely bypasses certain land acquisition challenges and provides a novel solution
	to eliminating bottlenecks in multimodal travel integration. The analyses performed in this paper
	are intended to optimize performance based on throughput and cruise conditions.
	Local transit presents different challenges and objectives than those seen in the longer distances
	considered in this study. This system model, however, could be easily adapted to perform
	analysis and optimization of local transit operation in the future.\\

\paragraph{Paper Scope}
	This paper introduces the first open-source system model to examine the
	full energy consumption and converged vehicle model for the Hyperloop concept.
	The full-system engineering model is built in Python to analyze the couplings
	between various sub-systems in the model, namely between the tube, passenger pod, and
	levitation systems. Using this model, trade studies were conducted to analyze the technological
	and economic feasibilities of the system.
	At the end, findings are presented regarding design aspects that simultaneously
	met all system constraints across each sub-discipline.

	The paper expands on previous works \cite{Chin} \textsuperscript{,}
	\cite{goodwin2009cantera}\textsuperscript{,} \cite{GrayBenchmarking2013}
	and is built using OpenMDAO, an open-source multidisciplinary analysis and
	optimization framework developed at the NASA Glenn Research Center.
	This work focuses on balancing the thermodynamic, aerodynamic, structural,
	weight, and power considerations of the system, with additional calculations for cost
	and mission design incorporated in the final system. Because each sub-system
	is designed independently of the whole, this model operates in a modular fashion, allowing
	for each sub-system the flexibility of being replaced with higher fidelity modules in the future.
	The model is not an attempt to obtain detail design conclusions for the Hyperloop
	-- rather, it is intended to perform sensitivity analyses
	and trade studies appropriate for evaluating the merit of the concept.

