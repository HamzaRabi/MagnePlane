\paragraph{Motivation}

    Currently, flying is the fastest and cheapest way for commercial
    passengers to travel long distances. However, for shorter routes the
    efficiency of air transport erodes quickly.
    The inherent upfront inefficiencies of air travel (e.g. arriving at the
    airport early, taxi time, climbing, descent, holding patterns)
    can take up more time than the actual flight.
    Additionally, many commercial aircraft are designed to accommodate much longer
    missions, and therefore, are not optimized for shorter trips.
    The Hyperloop is a new commercial transportation concept designed to offer
    both shorter travel times and lower ticket costs for short-haul routes.
    In order to achieve these reductions, Hyperloop merges a
    combination of aerospace and ground transportation technologies to achieve
    fundamentally new capability. The result is a mode of travel that is
    maximized both efficiency-wise and environmentally for short missions.

    The concept alters the traditional aviation mission profile by bringing
    the entire trajectory down to the ground level.
    As a result, a Hyperloop vehicle will not expend extra energy climbing
    and descending from a cruising altitude and can therefore spend more time
    operating at cruise conditions.
    Drag is reduced by flying through a low-pressure tube
    while conventional wings are replaced by alternative types of lifting bodies
    (e.g. magnetic and ground effect technologies) that are more compact and
    have lower induced drag than conventional aerodynamic surfaces.
    Traveling in a tube also allows the Hyperloop to operate independent of
    weather conditions and leverage newer and cleaner propulsion technologies.

    Since Hyperloop travels in a partial vacuum it must still contend with
    aerodynamic drag, though the magnitude of that drag is dramatically
    reduced. The challenge is the pressure accumulation as the passenger pod
    travels through the tube at transonic speed, causing increasing drag over time.
    To avoid the drag build up, the required tube diameter grows as the pod
    travel speed increases. At a design speed around Mach 0.8, the required tube
    size would be on the order of 4 meters in diameter. \cite{Chin}
    Travel tubes of this size will contribute
    substantially to the capital costs for constructing a Hyperloop system.
    In order to minimize the tube size, Hyperloop can employ an electric
    compressor in the front of the vehicle to prevent the build up
    of pressure in front of the vehicle. Existing aerospace research for
    electric aircraft propulsion can be leveraged to provide the necessary
    technology for this compressor along with recent technological developments
    in high-density electrical electronics and power storage. Aerospace modeling
    techniques also provide better tools for designing propulsions systems
    requiring boundary layer ingestion and management.

    Staying at ground-level allows the Hyperloop's electromagnetic propulsion
    system and much of the energy storage system to be located outside of the vehicle.
    This substantially reduces the weight and mechanical complexity of the
    individual pods compared to a traditional aircraft that must carry their
    own propulsion system and 100\% of the required mission fuel.
    There are a number of proposed methods for handling vehicle levitation and
    propulsion. The original Hyperloop Alpha proposal suggested the use of
    air-bearings for levitation.\cite{Musk}
    Subsequent efforts have focused more on magnetic levitation (MagLev) that
    can be more tightly coupled to the electromagnetic propulsion system for
    even higher efficiencies.
    The concept can be compatible with traditional MagLev systems developed for
    more conventional high-speed-rail (HSR) applications, but the partially
    evacuated tube provides an opportunity for even more efficient levitation.
    The substantially lower drag enables pulse and glide mission profiles,
    making passive MagLev more viable. This may result in a lighter, more affordable track
    system, avoiding active levitation components spanning the entire tube length.
    This unique combination of potential levitation and propulsion technologies,
    differentiate it from previous tube trains and MagLev systems.
    In the same way an airplane is differentiated from an airship or a hydroplane
    is differentiated from a boat, the Hyperloop is a plane that generates lift
    dynamically by moving over a medium at high-speeds
    rather than depending on buoyancy or static phenomena.
    The Hyperloop requires a fusion of technologies from both aerospace and HSR
    applications, though its transonic operation, air-breathing flow-path, and
    aerodynamically driven design give it qualities more akin to a plane than a train.
    The constrained, rarefied fluid flow around the vehicle
    fundamentally alters the design considerations necessary for Hyperloop and
    requires the consideration of many aerodynamic complexities that
    are common to aircraft, but not ever seen in train design.

    The entire system is held either above ground on concrete
    columns, underground or underwater within tunnels with the intent of
    maintaining a relatively straight trajectory.
    By sacrificing an airplane's traditional mission flexibility,
    Hyperloop can be optimized for passenger throughput, door-to-door travel time,
    and energy efficiency for open and level travel corridors spanning highly
    populous cities. The combination of these technologies allows the
    vehicle to spends more time operating under optimal conditions and avoids
    releasing carbon emissions into the upper atmosphere.

    The design mission is sized for distances between 250-500 nautical miles,
    which accounted for 57\% of commercial aircraft fleet operations in 2012.
    NASA's Strategic Implementation Plan stresses that global operations must
    keep pace with both an overall growing transportation market and
    simultaneous growth in market share dedicated to air transport.
    By 2050, 41\% of the world traffic (and 71\% of North America's) market share is
    projected to be high-speed transport. \cite{Schafer} This growth will be
    limited by the existing air-transportation infrastructure and will require
    dramatic improvements to air-traffic control capabilities to support
    increased numbers of flights in increasingly more congested air-space.

    The Hyperloop offers a compelling opportunity to offset this congestion by
    offering a faster and lower cost transportation option for a large portion
    of short-haul aviation routes. Although it requires substantial
    new infrastructure to be built, numerous high-population adjacent
    cities exist that have sufficient commercial travel volume to warrant the
    construction costs. These city pairs are often too far to conveniently
    travel by car and too short to efficiently travel by plane.
    Aeronautics market research shows that demand for this travel segment is
    the most sensitive to technology improvements and ticket price.\cite{Baik}
    The departure from typical aircraft constraints is intended to provide a
    compelling price point considering automobile vehicle miles are
    projected to increase by 7 billion simply due to airline
    fare increases by 10\% in 2015.

    Beyond the economic benefits of Hyperloop, it offers a significant
    reduction in carbon emissions. Planes produce large quantities of carbon emissions,
    exhausted into the upper atmosphere resulting in environmentally damaging travel.
    The fully-electric Hyperloop is designed for highly trafficked corridors,
    where its high upfront cost can be offset by cheaper, faster, and cleaner
    commercial travel on large scales.
    The Hyperloop is intended to alleviate billions of commuter car passenger miles,
    as well as free up airspace, reducing congestion and travel times for
    flights that are well suited for the national airspace system.\\


\paragraph{History}

    Aerospace engineers have promoted tube transport for over a century.
    As early as 1972, a study conducted by the RAND corporation concluded that
    high-speed `tubecraft' was technologically feasible, with political pressure being the greatest
    obstacle to creation.\cite{RAND} Many concepts later developed by
    the government, industry, and academia incorporated technologies from which
    Hyperloop can draw technical precedents.
    These include, but are not limited to, the Magneplane, Maglifter, and numerous
    other magnetic propulsion concepts. In 2013, Elon Musk, CEO of Space Exploration
    Technologies (SpaceX) and Tesla Motors revived the concept with the publication
    of his open-source paper, Hyperloop Alpha.\cite{Musk}
    Unlike previous waves of interest, this popularized design has spurred widespread international
    development efforts amongst leading universities, private companies with over
    \$100M in venture capitalist backing, and smaller research efforts at NASA and the
    United States Department of Transportation. \cite{Chin}

\paragraph{Focused Research Questions}

    The Hyperloop has largely evolved as an achievable concept over the past several years.
    In addition to Musk's paper, several key research questions were identified in a study conducted
    by the Department of Transportation (DOT) to quantify the potential value of the
    Hyperloop vehicle. \cite{Volpe} This paper works towards answering a few of
    those key drivers from both an engineering and business perspective.

    As highlighted by the DOT, the Hyperloop must be sufficiently lightweight
    that fixed capital costs of construction, when compared to conventional
    elevated HSR or Maglev systems, would be greatly decreased.
    Although true construction costs cannot be estimated through an engineering
    model alone, this paper seeks to minimize system size, energy, and material
    quantity as a proxy to cost. This model provides a sensitivity
    analysis of structural requirements as a function of pod mass to partially
    answer this research question. Rather than focus on weight of the system explicitly,
    many of the trends discussed are explored in relation to tube internal
    cross sectional area. Tube weight relates directly to this area term, and
    area is the common design variable linking many of the top-level
    analyses. As depicted in previous work,
    \cite{Chin} when abstracting the concept to three top level system level
    sizing variables (vehicle frontal area, speed, and tube cross-sectional area)
    only two are needed to find the third.

    Moreover, the potential ability to operate underwater is a key distinguishing feature
    of the Hyperloop versus conventional HSR and Maglev trains. It is possible for the buoyant force
    of the submerged tube to counter the systems weight, meaning weight is not
    a principle factor in this context.
    Given these considerations, the key research
    question can be restated as: ``Is the Hyperloop sufficiently compact so
    that there would be construction savings compared to conventional
    HSR or Maglev systems?'' Although you could put a conventional train in an
    underwater tube, these systems would not be capable of traveling nearly as
    fast as the Hyperloop in a volume constrained environment.
    Therefore subsequent results are presented in relation to the tube area.

    Another research query, highlighted by the DOT, questions the ability to
    expand the throughput of the system beyond the 28 passengers per pod
    originally proposed. This scaling capacity is addressed
    via analysis of pod frequency combined with design sensitivity to pod length.
    The relationship between pod mass and number of passengers per pod is also
    considered in subsequent structural analyses.

    The DOT also highlights the need to determine maximum payload capacities
    of the pods. Adding minimum required weights as model constraints is beyond the scope of
    this paper, but could be addressed using the engineering methods outlined.
    Answering this question and better defining component weight breakdowns
    is recommended as a next step for future work.
    This paper also does not address land acquisition costs or throughput
    constraints in today's existing transportation infrastructures.
    However, the cost modules in this model could consider these constraints if
    reliable data becomes available in the future.
    Capacity constrained water ports and airports are critical drivers referenced
    both in the DOT study and the NASA Aeronautics Strategic
    Implementation Plan. The prospect of high-speed underwater transportation
    completely bypasses certain land acquisition challenges and provides a novel solution
    to eliminating bottlenecks in multimodal travel integration. The analyses performed in this paper
    are intended to optimize performance based on throughput and cruise conditions.
    Local transit presents different challenges and objectives than those seen in the longer distances
    considered in this study. This system model, however, could be easily adapted to perform
    analysis and optimization of local transit operation in the future.\\

\paragraph{Paper Scope}
    This paper introduces the first open-source system model to examine the
    complete energy consumption and converged vehicle model for the Hyperloop concept.
    The full-system engineering model is built in Python to analyze the couplings
    between various subsystems in the model, namely between the tube, passenger pod, and
    levitation systems. Using this model, trade studies were conducted to analyze the technological
    and economic feasibilities of the system.
    At the end, findings are presented regarding design aspects that simultaneously
    met all system constraints across each sub-discipline.

    The paper expands on previous works \cite{Chin} and the model is built
    using OpenMDAO, an open-source multidisciplinary analysis and optimization
    framework written in Python and developed at the NASA Glenn Research Center.
    This work focuses on balancing the thermodynamic, aerodynamic, structural,
    weight, and power considerations of the system, with additional calculations for cost
    and mission design incorporated in the final system. Because each subsystem
    is designed independently of the whole, this model operates in a modular fashion, allowing
    for each subsystem the flexibility of being replaced with higher fidelity modules in the future.
    The model is not an attempt to obtain detail design conclusions for the Hyperloop
    -- rather, it is intended to perform sensitivity analyses
    and trade studies appropriate for evaluating the merit of the concept.

