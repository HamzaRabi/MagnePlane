\paragraph{Background of Hyperloop}

	Several key research questions were identified in a study conducted by the
	Department of Transportation to quantify the potential value of the
	Hyperloop concept. \cite{Volpe} This paper works towards answering a few of
	those key drivers from an engineering perspective:\\
	\\
	-Is the hyperloop transportation system sufficiently lightweight that there
	would significant construction cost savings compared to building an
	elevated HSR or maglev system?\\

	Although true construction costs cannot be estimated from an engineering
	model alone, this model seeks to minimize system size and material quantity
	as a proxy to cost. As it is posed, the question also implies
	that system weight is the most direct proxy to cost. This model provides a sensitivity
	analysis of structural requirements as a function of pod mass to partially
	answer this research question. Rather than focus on weight explicitly,
	many of the trends discussed are explored in relation to tube internal
	cross sectional area. Tube weight relates directly to this area term, and
	area is the common design variable linking many of the other top-level
	design variables. As depicted in previous work,
	\cite{Chin} when abstracting the concept to three top level system level
	sizing variables;(vehicle area, speed, and tube area) only two are needed
	to find the third.

	The potential ability to operate underwater is a key distinguishing feature
	of Hyperloop over conventional HSR and maglev, and in this context, weight
	isn't a principle factor. Given these considerations, the key research
	question can be rephrased to: ``Is the hyperloop sufficiently compact that
	there would be construction savings compared to conventional
	HSR or maglev systems?'' Many of the following results are therefore posed
	in relation to tube area.\\
	\\
	-Can the capacity of the hyperloop pod be expanded to seat more than the
	originally proposed 28 passengers?\\

	Given a cost, throughput efficiency must be evaluated to quantify the
	Hyperloop's value as a transportation mode.
	Capacity is addressed via analysis of pod frequency combined with design
	sensitivity to pod length.\\

	-How big would the tube need to be in order to
	carry a standard size shipping container?\\

	-What would be the [payload] weight limit for a freight capsule?\\

	This work focuses on setting up the relationships between subsystems
	necessary for a system level optimization problem. Adding these two research
	drivers in the form of minimum constraints is beyond the scope of this paper
	but could be addressed using the engineering methods outlined in this paper.
	Answering these questions is recommended as next steps for future work.\\

	- What are the technology hurdles to building a Hyperloop underwater and can
	they be overcome?\\

	- Can the system be designed so that in addition to carrying long distance
	passengers, it can also provide local transit service?\\

	This paper does not address land aquisition costs, or throughput
	constraints in today's existing transportation infrastrucutre.
	Capacity constrained water ports and airports are critical driver referenced
	both in the DOT study, and a key focus of NASA Aeronautics Strategic
	Implementation Plan. The prospect of high-speed underwater transportation
	completely bypasses land aquisition challenges, and provides a novel solution
	to eliminate bottlenecks in multimodal travel integration.
	Although not addressed in the following engineering models,
	stressing the importance of pursuing underwater
	Hyperloop capabilities is further reiterated.\\


	-Shortcomings of short-haul aviation\\
	-Shortcomings of high speed rail project / driving\\


\paragraph{History of Hyperloop}

	Aerospace engineers have promoted tube transport over the course of a century;
	the most prominent include Robert Goddard \cite{Goddard} (creator of the first liquid fueld
	rocket) to Dr. Robert Salter (a forefather of the satellite). As early as 1972,
	a study conducted by the RAND coperation concluded that high-speed `tubecraft'
	was technologically feasible with political pressure being the greatest
	obstacle.\cite{RAND} National interest was once again rekindled in 2013 with a refreshed
	Hyperloop concept championed by Elon Musk, CEO of Space Exploration
	Technologies (SpaceX) and Tesla Motors.\cite{Musk}
	Unlike previous waves of interest, the currently
	popularized design has spurred widespread international development efforts
	amongst hundreds of leading universities, private companies with over \$100M
	in venture capitalist backing, and smaller research efforts at NASA and the
	US Department of Transportation. \cite{Chin}
	The design has continously evolved, with the latest Hyperloop derivative
	generating lift using magnetic levitation.
	In the same way an airplane is differentiated from an airship,
	or a hydroplane is differentiated from a boat,
	Magneplane is a plane in the sense that it generates lift dynamically
	by moving over a medium at high-speeds,
	rather than depending on buoyancy or static phenomena.
	It's transonic operation, air-breathing flow-path and aerodynamic
	driven design give it qualities closer to a plane rather than a train.
	The concept deviates from existing high-speed rail designs by eliminating
	the rails, enclosing the passenger pod in a tube under a partial vacuum,
	with propulsion handled by a set of linear electromagnetic accelerators
	mounted to the tube. The entire system is held either above ground on concrete
	columns or subterranian tunnels maintaining a relatively straight trajectory.

	Hyperloop flies low enough that propulsion can be offloaded to
	ground systems, intended to rival efficiencies previously limited to terrestrial
	vehicles. By sacrificing an airplane’s traditional mission flexibility,
	it’s aimed to optimize for passenger throughput, door-to-door travel time, and energy
	efficiency.

	As a first step, the concept is sized for distances between 250-500 nautical miles,
	which comprised 57\% of commercial aircraft fleet operations in 2012.

	As stressed in NASA's Strategic Implementation Plan, operations must keep pace
	with both an overall growing transportation market and simultaneous growth in
	market share dedicated to high-speed transport. By 2050, 41\% of world traffic
	market share will be high-speed transport. \cite{Schafer}

	Hyperloop offers a compelling opportunity to offset this congestion.
	In terms of travel speed and door-to-door travel time,
	it's designed to occupy a highly desired gap in travel distance, where it's
	too long to conveniently travel by ground options,
	but too short to make the upfront delays of air travel worthwhile.

	Aeronautics market research \cite{H. Baik} also shows that demand for this
	travel segment is the most sensitive to technology improvements and ticket
	price. The departure from typical aircraft constraints
	is intended to provide a compelling price point
	considering automobile vehicle miles were
	projected to increase by 7 billion simply due to airline
	fare increases by 10\% in 2015.

\paragraph{How Hyperloop solves problem}
	-Cleaner, faster, cheaper, bitch!\\
	-High level technological overview\\

	More than half of today’s commercial flights are less than 500 nm
	and today’s airplanes are not at all suited for this type of mission.
	On these flights airplanes spend very little time in the cruise segment that they’re optimized for.
	They spend a significant amount of time not even flying,
	as they are slow to board, deboard, as well as taxi-ing and de-icing, nevermind airport security.
	Even once they’re in the air they waste time and energy on: take-off, climb,
	holding patterns, descent and landing, which are all sub-optimal conditions for the vehicle.

	The entire system is also very susceptible to inclement weather.
	The Hyperloop concept throws out all of these efficiencies by operating at
	optimal conditions for the entire mission by bringing the entire trajectory down to ground level.
	Drag is reduced by flying through a low-pressure tube,
	and conventional wings are replaced by alternative types of lifting bodies
	that are more compact with comparatively lower induced drag.
	While a conventional plane wastes a lot of energy getting up,
	and coming back down from 35,000ft,
	flying close to the ground obviously bypasses these inefficiencies entirely
	and also has numerous other subtle benefits.
	It allows the propulsion system to be located off-board from the vehicle itself.
	This allows the vehicle to be dramatically lighter and less complex than a
	counterpart aircraft of the same passenger throughput.
	Similarly the fully electric vehicle isn’t burdened by the need to carry it’s own fuel,
	saving on both weight and again, complexity.
	And no harmful emissions are released into the upper atmosphere.

	The concept has serious potential to alleviate billions of commuter car passenger miles,
	as well as free up airspace, reducing congestion and travel times
	for flights that are well suited for the national airspace system.

\paragraph{Layout about what this paper addresses}
	-Full system model in OpenMDAO\\
	--Technological feasibility\\
	--Economic feasibility\\
	-Trade Studies\\
	-Conclusions\\

	The model expands on previous works \cite{Chin} \textsuperscript{,}
	\cite{goodwin2009cantera}\textsuperscript{,} \cite{GrayBenchmarking2013}
	and is built using OpenMDAO, a python-based modeling and optimizaiton framework.
	This work focuses on balancing the thermodynamic, aerodnymanic, structural,
	weight, and power considerations, with additional calculations for cost
	and mission design.

	-discuss importance of simulatenous design of multi-discp subsystems
	-emphasize broad connectivity of the model as it's strength
	(as opposed to subsystem fidelity)


