\paragraph{History}

	The Hyperloop is a high speed vehicle concept with the potential to revolutionize future
	transportation. Aerospace engineers have promoted tube transport over the course of the 
	last century. As early as 1972, a study conducted by the RAND corporation concluded that high-
	speed `tubecraft' was technologically feasible, with political pressure being the greatest
	obstacle to creation.\cite{RAND} In 2013, Elon Musk, CEO of Space Exploration
	Technologies (SpaceX) and Tesla Motors.\cite{Musk} revived the concept with the publication
	of his open-source paper, Hyperloop Alpha.\cite{Musk}
	Unlike previous waves of interest, this popularized design has spurred widespread international 
	development efforts amongst leading universities, private companies with over \
	$100M in venture capitalist backing, and smaller research efforts at NASA and the
	United States Department of Transportation. \cite{Chin} The Hyperloop has serious potential to 
	alleviate billions of commuter car passenger miles, as well as free up airspace, reducing 
	congestion and travel times for flights that are well suited for the national airspace system.
	
	The design of the Hyperloop has continuously evolved, with the latest derivative
	generating lift using magnetic levitation, much like those in Maglev trains.
	In the same way an airplane is differentiated from an airship or a hydroplane is differentiated from 
	a boat, the Hyperloop is a plane that generates lift dynamically
	by moving over a medium at high-speeds,
	rather than depending on buoyancy or static phenomena.
	Its transonic operation, air-breathing flow-path and aerodynamic
	driven design give it qualities closer to a plane rather than a train.
	The concept deviates from existing high-speed rail designs by eliminating
	the rails, enclosing the passenger pod in a tube under a partial vacuum,
	with propulsion handled by a set of linear electromagnetic accelerators
	mounted to the tube. The entire system is held either above ground on concrete
	columns or sub-terranian tunnels maintaining a relatively straight trajectory. The Hyperloop flies 
	low enough that propulsion can be offloaded to ground systems, intended to rival efficiencies 
	previously limited to terrestrial vehicles. By sacrificing an airplane'��s traditional mission flexibility,
	it'��s aimed to optimize for passenger throughput, door-to-door travel time, and energy
	efficiency.

	As a first step, the concept is sized for distances between 250-500 nautical miles,
	which comprised 57\% of commercial aircraft fleet operations in 2012. As stressed in NASA's 
	Strategic Implementation Plan, global operations must keep pace with both an overall growing 
	transportation market and simultaneous growth in market share dedicated to high-speed transport. 
	By 2050, 41\% of world traffic market share will be high-speed transport. \cite{Schafer}

	The Hyperloop offers a compelling opportunity to offset this congestion.
	It is designed to occupy a highly desired gap in travel distance and traverse a distance
	that is currently too far to travel by car and too short to travel by plane. Aeronautics market 
	research \cite{H. Baik} also shows that demand for this travel segment is the most sensitive to 
	technology improvements and ticket price. The departure from typical aircraft constraints
	is intended to provide a compelling price point considering automobile vehicle miles are
	projected to increase by 7 billion simply due to airline
	fare increases by 10\% in 2015.

\paragraph{Background}

	The Hyperloop has largely evolved as an achievable concept over the past several years.
	In addition to Musk's paper, several key research questions were identified in a study conducted 
	by the Department of Transportation to quantify the potential value of the
	Hyperloop vehicle. \cite{Volpe} This paper works towards answering a few of
	those key drivers from both an engineering and business perspective:\\
	
	\\
	-Why do we need the Hyperloop vehicle concept?\\
	
	Today, existing modes of transportation are woefully inept at providing efficient, cheap, and
	clean travel. More than 50 percent of current aircraft operations serve flights under 500
	miles, with the majority of the time spent on the ground. When analyzing commercial
	travel options for distances between 200-500 miles, such as the highly trafficked route of LA-SF, 
	the only options are by car or by plane. Neither option are efficient nor cheap. Furthermore,
	the carbon emissions produced by cars and planes are plentiful, resulting in not only
	time-costly, but also environmentally damaging travel. The Hyperloop is designed to 
	optimize for distances in this range by providing a cheaper, faster, and cleaner 
	option for commercial travel.
	
	\\
	-Is the Hyperloop sufficiently lightweight that fixed capital costs of construction
	when compared to elevated HSR or Maglev systems would be greatly decreased?\\

	Although true construction costs cannot be estimated through an engineering
	model alone, this paper seeks to minimize system size and material quantity
	as a proxy to cost. This model provides a sensitivity
	analysis of structural requirements as a function of pod mass to partially
	answer this research question. Rather than focus on weight of the system explicitly,
	many of the trends discussed are explored in relation to tube internal
	cross sectional area. Tube weight relates directly to this area term, and
	area is the common design variable linking many of the other top-level
	design variables. As depicted in previous work,
	\cite{Chin} when abstracting the concept to three top level system level
	sizing variables;(vehicle area, speed, and tube area) only two are needed
	to find the third.

	Moreover, the potential ability to operate underwater is a key distinguishing feature
	of the Hyperloop over conventional HSR and Maglev trains. in this context, weight
	isn't a principle factor. Given these considerations, the key research
	question can be restated as: ``Is the Hyperloop sufficiently compact that
	there would be construction savings compared to conventional
	HSR or Maglev systems?'' We will present many results in relation to the tube area.\\
	\\
	-Can the capacity of the Hyperloop pod be expanded to seat more than the 28
	passengers originally proposed by Musk?\\

	Given a cost, throughput efficiency of the system must be evaluated to quantify the
	Hyperloop's value as a transportation vehicle. This capacity is addressed via analysis of pod 
	frequency combined with design sensitivity to pod length.\\

	-What would be the [payload] weight limit for a freight capsule?\\

	This work focuses on analyzing the relationships between sub-systems
	necessary for a system level optimization problem. Adding these two research
	drivers in the form of minimum constraints is beyond the scope of this paper
	but could be addressed using the engineering methods outlined in this paper.
	Answering these questions is recommended as next steps for future work.\\

	- Can the system be designed so that, in addition to carrying long distance
	passengers, it can also provide local transit service?\\

	This paper does not address land acquisition costs or throughput
	constraints in today's existing transportation infrastructures.
	Capacity constrained water ports and airports are critical drivers referenced
	both in the DOT study and the NASA Aeronautics Strategic
	Implementation Plan. The prospect of high-speed underwater transportation
	completely bypasses land acquisition challenges and provides a novel solution
	to eliminating bottlenecks in multimodal travel integration.\\

\paragraph{Overview of the System}

	As referenced above, today's aircraft are not suited for the majority of flights operated.
	Short-haul airplanes spend very little time in the cruise segment that they'��re optimized for.
	They spend a significant amount of time not even flying,
	as they are slow to board, de-board, as well as taxi-ing and de-icing.
	Even once they're in the air time and energy is wasted on the take-off, climb,
	holding pattern, descent and landing segments of the flight, which all operate
	sub-optimally.

	Furthermore, aircraft is also extremely susceptible to inclement weather.
	The Hyperloop concept optimizes all of these inefficiencies and operates at
	optimal conditions by bringing the entire trajectory down to ground level.
	Drag is reduced by flying through a low-pressure tube
	and conventional wings are replaced by alternative types of lifting bodies
	that are more compact with comparatively lower induced drag.
	While a conventional plane wastes a lot of energy getting up,
	and coming back down from 35,000ft,
	flying close to the ground obviously bypasses these inefficiencies entirely
	and also has numerous other subtle benefits.
	The propulsion system can be be located off-board from the vehicle itself, 
	allowing the vehicle to be dramatically lighter and less mechanically complex than a
	traditional aircraft with the same passenger throughput.
	Furthermore, the fully electric vehicle isn't� burdened by the need to carry its own fuel,
	resulting in savings of both weight and fuel. No carbon emissions are released, 
	resulting in a completely clean vehicle.

\paragraph{What this Paper Addresses}
	This paper will introduce the first full-system model ever created for the Hyperloop. A full-system
	engineering model was built in Python to analyze the couplings 
	between various sub-systems in the model, such as between the tube, passenger pod, and 
	levitation systems. Using this model, trade studies were conducted to analyze the technological 
	and economic feasibilities of the system. At the end, the paper will present findings regarding 
	optimal structures and engineering design.
	
	The paper expands on previous works \cite{Chin} \textsuperscript{,}
	\cite{goodwin2009cantera}\textsuperscript{,} \cite{GrayBenchmarking2013}
	and is built using OpenMDAO, an in-house NASA modeling and optimization framework.
	This work focuses on balancing the thermodynamic, aerodynamic, structural,
	weight, and power considerations of the system, with additional calculations for cost
	and mission design incorporated in the final system. Because each sub-system
	is designed independently of the whole, this model operates in a modular fashion, allowing
	for each sub-system the flexibility of being replaced with a higher fidelity module. Our system
	is not an attempt to derive niche conclusions regarding the engineering design of the Hyperloop--
	rather, it is intended to explore the trends observed throughout the course of technological 
	design.

