\paragraph{Motivation}

	Currently, the most comomon method for high-speed travel over long distances is air travel.
	To reduce operating costs, commercial airliners often focus their efforts on optimizing aircraft for
	cruise conditions since this is the largest segment of travel over long distances.
	However, when traveling shorter distances (under 500 miles), vehicles do not spend a majority of their travel time in cruise.
	Instead, the time spent warming up, taxiing, taking off, climbing, descending, and landing
	takes up a larger portion of the total travel time. In general, short-haul aircraft are not optimized
	for these phases of travel, which reduces their overall efficiency. The result is a mode of travel that is
	unnecessarily costly and environmentally damaging.

	The Hyperloop is a high speed vehicle concept designed to meet the challenges facing the short-haul aviation industry.
	The concept alters the mission profile of traditional aviation by bringing the entire trajectory down to the ground level.
	As a result, a Hyperloop vehicle will not have to expend energy climbing and descending from
	a cruising altitude and can instead spend more time operating at cruise conditions.
	Drag is reduced by flying through a low-pressure tube
	and conventional wings are replaced by alternative types of lifting bodies
	that are more compact and have lower induced drag than conventional aerodynamic surfaces.
	Traveling in a tube allows the the Hyperloop to leverage newer, cleaner propulsion technologies.
	An electromagnetic propulsion system can be located outside of the vehicle,
	which reduces the weight and mechanical complexity of the Hyperloop vehicle compared to a
	traditional aircraft with the same passenger throughput.
	The Hyperloop vehicle can also include a fully electric compressor in the front of the vehicle to
	further reduce drag by preventing the build up of pressure in front of the vehicle.
	The compressor can be fully electric, allowing the Hyperloop to infuse recent technological develpments in
	high-density electrical power storage and fully electric turbomachinery.
	The result is a technologically advanced vehicle that spends more time operating under optimal conditions
	all while producing zero carbon emissions.

	The Hyperloop is intended to alleviate billions of commuter car passenger miles,
	as well as free up airspace, reducing congestion and travel times for
	flights that are well suited for the national airspace system.

	The design of the Hyperloop has continuously evolved, with the latest derivative
	generating lift using a specialized magnetic levitation (MagLev) system.
	In the same way an airplane is differentiated from an airship or a hydroplane is differentiated from
	a boat, the Hyperloop is a plane that generates lift dynamically
	by moving over a medium at high-speeds
	rather than depending on buoyancy or static phenomena.
	Its transonic operation, air-breathing flow-path and aerodynamic
	driven design give it qualities more similar to a plane rather than a train.
	The concept deviates from existing high-speed rail designs by enclosing the passenger pod in a tube under a partial vacuum
	propelled by a set of linear electromagnetic accelerators
	mounted to the tube. The entire system is held either above ground on concrete
	columns, underground in sub-terranian tunnels, or in underwater tunnels while maintaining a relatively straight trajectory.
	The Hyperloop flies low enough that propulsion can be offloaded to ground systems,
	intended to rival efficiencies previously limited to terrestrial vehicles.
	By sacrificing an airplane's traditional mission flexibility,
	Hyperloop can be optimized for passenger throughput, door-to-door travel time,
	and energy efficiency.
	This unique combinations of potential levitation and propulsion technologies,
	differentiate it from previous tube trains and maglev systems.

	The concept is sized for distances between 250-500 nautical miles,
	which accounts for 57\% of commercial aircraft fleet operations in 2012. NASA's
	Strategic Implementation Plan stresses that global operations must keep pace with both an overall growing
	transportation market and simultaneous growth in market share dedicated to high-speed transport.
	By 2050, 41\% of world traffic market share is projected to be high-speed transport. \cite{Schafer}

	The Hyperloop offers a compelling opportunity to offset this congestion.
	It is designed to occupy a highly desired gap in travel distance and traverse a distance
	that is currently too far to travel by car and too short to travel by plane. Aeronautics market
	research also shows that demand for this travel segment is the most sensitive to
	technology improvements and ticket price.\cite{H. Baik} The departure from typical aircraft constraints
	is intended to provide a compelling price point considering automobile vehicle miles are
	projected to increase by 7 billion simply due to airline
	fare increases by 10\% in 2015.

\paragraph{History}

	Aerospace engineers have promoted tube transport for over a century.
	As early as 1972, a study conducted by the RAND corporation concluded that high-
	speed `tubecraft' was technologically feasible, with political pressure being the greatest
	obstacle to creation.\cite{RAND} Many concepts later developed by
	the government, industry, and academia incorporated technologies from which
	Hyperloop can draw technical precedents.
	These include but are not limited to the Magneplane, Maglifter, and numerous
	other magnetic propulsion concepts. In 2013, Elon Musk, CEO of Space Exploration
	Technologies (SpaceX) and Tesla Motors revived the concept with the publication
	of his open-source paper, Hyperloop Alpha.\cite{Musk}
	Unlike previous waves of interest, this popularized design has spurred widespread international
	development efforts amongst leading universities, private companies with over
	\$100M in venture capitalist backing, and smaller research efforts at NASA and the
	United States Department of Transportation. \cite{Chin}

\paragraph{Focused Research Questions}

	The Hyperloop has largely evolved as an achievable concept over the past several years.
	In addition to Musk's paper, several key research questions were identified in a study conducted
	by the Department of Transportation to quantify the potential value of the
	Hyperloop vehicle. \cite{Volpe} This paper works towards answering a few of
	those key drivers from both an engineering and business perspective:\\

	\\
	-Why do we need the Hyperloop vehicle concept?\\

	Today, existing modes of transportation are woefully inept at providing efficient, cheap, and
	clean travel. More than 50 percent of current aircraft operations serve flights under 500
	miles, with the majority of the time spent on the ground. When analyzing commercial
	travel options for distances between 250-500 miles, such as the highly trafficked route of LA-SF,
	the only options are by car or by plane. These options are neither efficient nor cheap. Furthermore,
	both cars and planes produce large quantities of carbon emissions, resulting in not only
	time-costly, but also environmentally damaging travel. The Hyperloop is designed to be
	optimal for distances in this range by providing a cheaper, faster, and cleaner
	option for commercial travel.\\

	\\
	-Is the Hyperloop sufficiently lightweight that fixed capital costs of construction
	when compared to elevated HSR or Maglev systems would be greatly decreased?\\

	Although true construction costs cannot be estimated through an engineering
	model alone, this paper seeks to minimize system size, energy, and material
	quantity as a proxy to cost. This model provides a sensitivity
	analysis of structural requirements as a function of pod mass to partially
	answer this research question. Rather than focus on weight of the system explicitly,
	many of the trends discussed are explored in relation to tube internal
	cross sectional area. Tube weight relates directly to this area term, and
	area is the common design variable linking many of the other top-level
	design variables. As depicted in previous work,
	\cite{Chin} when abstracting the concept to three top level system level
	sizing variables;(vehicle area, speed, and tube area) only two are needed
	to find the third.

	Moreover, the potential ability to operate underwater is a key distinguishing feature
	of the Hyperloop over conventional HSR and Maglev trains. It is possible for the buoyant force
	of the submerged tube to counter the systems weight, meaning weiht is not a principle factor in this context.
	Given these considerations, the key research
	question can be restated as: ``Is the Hyperloop sufficiently compact that
	there would be construction savings compared to conventional
	HSR or Maglev systems?'' We will present many results in relation to the tube area.\\
	\\
	-Can the capacity of the Hyperloop pod be expanded to seat more than the 28
	passengers originally proposed by Musk?\\

	Given a cost, throughput efficiency of the system must be evaluated to quantify the
	Hyperloop's value as a transportation vehicle. This capacity is addressed
	via analysis of pod frequency combined with design sensitivity to pod length.
	The relationship between pod mass and number of passengers per pod is also considered
	in structural analysis.\\

	-What would be the [payload] weight limit for a freight capsule?\\

	This work focuses on analyzing the relationships between sub-systems
	necessary for a system level optimization problem. Adding these two research
	drivers in the form of minimum constraints is beyond the scope of this paper
	but could be addressed using the engineering methods outlined in this paper.
	Answering these questions is recommended as next steps for future work.\\

	- Can the system be designed so that, in addition to carrying long distance
	passengers, it can also provide local transit service?\\

	This paper does not address land acquisition costs or throughput
	constraints in today's existing transportation infrastructures. However, the cost
	modules in this model could consider these constraints if reliable data becomes available in the future.
	Capacity constrained water ports and airports are critical drivers referenced
	both in the DOT study and the NASA Aeronautics Strategic
	Implementation Plan. The prospect of high-speed underwater transportation
	completely bypasses certain land acquisition challenges and provides a novel solution
	to eliminating bottlenecks in multimodal travel integration. The analyses performed in this paper
	are intended to optimize performace based on throughput and cruise conditions.
	Local transit presents different challenges and objectives than those seen in the longer distances
	considered in this study. This system model, however, could be easily adapted to perform
	analysis and optimzation of local transit operation in the future.\\

\paragraph{Paper Scope}
	This paper introduces the first open-source system model to examine the
	full energy consumption and converged vehicle model for the Hyperloop concept.
	The full-system engineering model is built in Python to analyze the couplings
	between various sub-systems in the model, namely between the tube, passenger pod, and
	levitation systems. Using this model, trade studies were conducted to analyze the technological
	and economic feasibilities of the system.
	At the end, findings are presented regarding design aspects that
	met all system constraints.

	The paper expands on previous works \cite{Chin} \textsuperscript{,}
	\cite{goodwin2009cantera}\textsuperscript{,} \cite{GrayBenchmarking2013}
	and is built using OpenMDAO, an open-source multidisciplinary analysis and optimization framework
	developed at the NASA Glenn Research Center.
	This work focuses on balancing the thermodynamic, aerodynamic, structural,
	weight, and power considerations of the system, with additional calculations for cost
	and mission design incorporated in the final system. Because each sub-system
	is designed independently of the whole, this model operates in a modular fashion, allowing
	for each sub-system the flexibility of being replaced with higher fidelity modules in the future.
	The model is not an attempt to obtain detail design conclusions regarding the
	engineering design of the Hyperloop -- rather, it is intended to perform sensitivity analyses and
	trade studies appropriate for the conceptual design phase.

