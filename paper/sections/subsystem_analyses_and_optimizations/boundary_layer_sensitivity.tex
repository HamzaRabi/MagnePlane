In this system model, the tube cross section is sized such that the flow in the pod does not choke. Previous models attempted to impose this condition by assuming that the area of the bypass flow is equal to the tube cross sectional area minus the pod cross sectional area (cite NASA paper). This model attempts to build upon previous models by also accounting for the development of a boundary layer over the pod surface. The boundary layer formation further reduces the bypass area and increases the chance of choking in the flow. Thus, a larger tube will be necessary to prevent choking when boundary layer effects are accounted for, which will increase the costs of both raw materials and construction. This model will first analyze the sensitivity of the tube cross sectional area to the boundary layer growth along the pod surface. Then, estimations of boundary layer thickness as a function of length-based $Re$ will be made in order to analyze the sensitivity of tube cross sectional area to changes in pod length.
For each analysis, the tube is sized using the previously discussed inviscid, quasi-1D area relationships for compressible flow. To account for the boundary layer using this method, the boundary layer is modeled as a cylindrical ring with a thickness equal to the maximum thickness of the displacement boundary layer,  $delta_star$, over the outside of the pod, effectively reducing the area of the bypass flow. Using this model, the contraction of the bypass flow is given by the equation
\begin{equation}
	\label{eq:epsilon}
	Define epsilon equation
\end{equation}
This relationship accounts for the reduction in bypass area due to a boundary layer of arbitrary displacement thickness. First, the sensitivity of tube area to displacement boundary layer thickness is studied over a range of boundary layer thicknesses for various pod cross sections. Next, the model uses a flat plate approximation to obtain a relationship between boundary layer and length. Since $Re > 500,000$, the boundary layer is assumed turbulent with a velocity profile \cite{FoxAndMcDonald}
\begin{equation}
	\label{eq:boundary_layer_profile}
	Boundary Layer velocity profile equation
\end{equation}
Using this boundary layer velocity profile, the thickness of the displacement boundary layer is derived using a similarity solution to produce the equation \cite{FoxAndMcDonald}
\begin{equation}
	\label{eq:boundary_layer}
	Displacement Boundary Layer equation
\end{equation}
The system model uses there to relate displacement boundary layer thickness, and therefore tube area, with pod length. It is understood that these equations do not accurately represent the physical relationships in our model because they are derived in the textbook by Fox and McDonald assuming 1D flow over a flat plate with zero pressure gradient. Characterizing the displacement thickness of the boundary layer in 3D conical flow is beyond the scope of this analysis. However, the approximation made in this analysis is useful for studying how tube area and material costs may change with increasing pod length within an order of magnitude of accuracy.  The pod configurations tested in each analysis is given in $Insert Table Number$.
\begin{table}
	\centering
	\caption{Configuratoins in boundary layer sensitivity study}
	\label{tbl:boundary_layer_sensitivty_configs}
	% \begin{tabular}
	% \end{tabular}
\end{table}
\begin{figure}
	\centering
	\caption{Tube Cross Sectional Area vs. Displacement Boundary Layer Thickness and Pod Cross Sectional Area}
	\label{fig:cross_sec_area_vs_disp_boundary_layer}
	\includegraphics{../../images/cross_sec_area_vs_disp_bound_layer.png}
\end{figure}
\Cref{fig:cross_sec_area_vs_disp_boundary_layer} indicated the effects that displacement boundary layer has on the cross sectional area of the tube. It is shown that the cross sectional area of the tube increases rapidly as the thickness of the displacement boundary layer increases. For $A_pod = \SI{2d2}{\m\squared}$, a 1 cm increase in $delta*$ results in an increase of approximately 1.57 m^2 in $A_tube$. Thus, it can be inferred that the tube area, and therefore the tube material cost, are extremely sensitive to boundary layer thickness. Furthermore, the rate of increase of the tube cross sectional area also increases with pod cross sectional area. This relationship exists because more bypass is area is lost as pod radius increases for a given displacement boundary layer thickness. This means that the coupling between tube and pod cross sectional areas is even stronger than was indicated in previous research.
\begin{figure}
	\centering
	\caption{Tube Cross Sectional Area vs. Pod Length}
	\label{fig:tube_area_vs_length}
	\includegraphics{../../images/tube_area_vs_length.png}
\end{figure}
\Cref{fig:tube_area_vs_oength} shows the necessary cross sectional area of the tube as a function of pod length using the previously described relationship approximation. It is shown that increases in pod length due to higher passenger capacity, increased battery length, more compressor stages, etc. require the tube cross sectional area to increase in order to accommodate for larger boundary layer growth. This relationship reveals a coupling between tube cross sectional area and pod length that has not been accounted for in previous system models.
The tube area is highly coupled with boundary layer thickness due to its effect on effective bypass area. Due to the sensitivity of tube area with respect to boundary layer thickness, it is important to develop a model that can accurately determine the displacement boundary layer thickness over the pod surface in order to avoid undesired choking or development of shock waves in the tube. Moreover, this analysis suggests that active flow techniques, such as boundary layer suction, which reduce boundary layer thickness could be very beneficial. Since the lowest tube cross sectional area is desired for a given pod configuration, boundary layer suction could reduce the size of the tube necessary to prevent choking which will reduce the total material and construction costs of the system.
